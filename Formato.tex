\usepackage[utf8]{inputenc}
\usepackage[spanish]{babel}
%%%%%%%%%%%%%%%%%%%%%%%%%%%%%%%%%%%%%%%%%%%%%%%%%%%%%%%%%%%%%%%%%%%%%%%%%%%%%%%%%%%%%%%%%%%%%%%%%%%%%%%
\usepackage{afterpage}% Necesario para introdicur páginas A3
\usepackage{amsmath,amsthm,amstext,amssymb}
\usepackage{capt-of}
\usepackage{colortbl}
\usepackage{graphicx}% Permite la introducción de figuras
\usepackage{array,fancyhdr,graphicx,subfigure,titlesec,titletoc,xcolor}
\usepackage{emptypage}% evita la numeración de las páginas en blanco
\usepackage{etoolbox}
\usepackage{listings} % premite la introducción de códigos vhdl
\usepackage{minted}
\usepackage[paper=A4,pagesize]{typearea}%necesario para introducir páginas A3
\usepackage{lscape}% Necesario para páginas apaisadas
\usepackage{pdfpages}% Permite introducir documentos pdf
\usepackage[font=small,bf]{caption}% Formato del caption
\usepackage{rotating}% Rotaciones
\usepackage{setspace,subfigure}
\usepackage{tocstyle}
\usepackage{codigomatlab}
\usepackage{titlesec}
\usepackage{color}
\usepackage{tikz}
\usetikzlibrary{calc}
\usepackage{pstricks}
\usepackage{pst-node}
\usepackage{pst-blur}
\usepackage{amsmath,amssymb}
\usepackage{booktabs}
\usepackage{xcolor}
\usepackage{colortbl}
\usepackage{rotating}
\usepackage{multirow}
\usepackage{bigstrut}
\usepackage{eurosym}
%\usepackage{inputenc}
%%%%%%%%%%%%%%%%%%%%%%%%%%%%%%%%%%%%%%%%%%%%%%%%%%%%%%%%%%%%%%%%%%%%%%%%%%%%%%%%%%%%%%%%%%%
\newcommand{\documento}{ÍÍNDICE XERAL}
%%%%%%%%%%%%%%%%%%%%%%%%%%%%%%%%%%%%%%%%%%%%%%%%%%%%%%%%%%%%%%%%%%%%%%%%%%%%%%%%%%%%%%%%%%%
% Incluye la bibliografía como sección
\makeatletter
\renewenvironment{thebibliography}[1]
     {\chapter{\bibname}% esta línea cambia la bibliografía a la categoría sección
      \@mkboth{\MakeUppercase\bibname}{\MakeUppercase\bibname}
      \list{\@biblabel{\@arabic\c@enumiv}}%
           {\settowidth\labelwidth{\@biblabel{#1}}
            \leftmargin\labelwidth
            \advance\leftmargin\labelsep
            \@openbib@code
            \usecounter{enumiv}
            \let\p@enumiv\@empty
            \renewcommand\theenumiv{\@arabic\c@enumiv}}
      \sloppy
      \clubpenalty4000
      \@clubpenalty \clubpenalty
      \widowpenalty4000%
      \sfcode`\.\@m}
     {\def\@noitemerr
       {\@latex@warning{Empty `thebibliography' environment}}
      \endlist}
\makeatother
%%%%%%%%%%%%%%%%%%%%%%%%%%%%%%%%%%%%%%%%%%%%%%%%%%%%%%%%%%%%%%%%%%%%%%%%%%%%%%%%%%%%%%%%%%%%%%%%%%%%%%%%%%%%%%%%%%%%%%%%%%%%%%%%%%%%%%%%%%%%%%%%%%%%%%%%%%%%%%%%%%%%%
%FORMATO DE LA HOJA
\renewcommand{\baselinestretch}{1.25}
\headsep 8mm        \topmargin -1.5cm     \textheight 24.5cm     \textwidth 16cm     \oddsidemargin 0.5cm     \evensidemargin -0.1cm
 \footnotesep=20pt               \footskip=38pt
%%%%%%%%%%%%%%%%%%%%%%%%%%%%%%%%%%%%%%%%%%%%%%%%%%%%%%%%%%%%%%%%%%%%%%%%%%%%%%%%%%%%%%%%%%%%%%%%%%%%%%%%%%%%%%%%%%%%%%%%%%%%%%%%%%%%%%%%%%%%%%%%%%%%%%%%%%%%%%%%%%%%%
% Formato del título de cada parte
\titleformat{\part}[display]
  {\normalfont\huge\bfseries}
	{}{0pt}{\centering}
	
\titlespacing*{\part}{0pt}{0pt}{20pt}
\titleclass{\part}{straight}
%%%%%%%%%%%%%%%%%%%%%%%%%%%%%%%%%%%%%%%%%%%%%%%%%%%%%%%%%%%%%%%%%%%%%%%%
%Formato del título de cada capítulo
\titleformat{\chapter}[hang] 
{\normalfont\huge\bfseries}{\thechapter}{1em}{} 
%%%%%%%%%%%%%%%%%%%%%%%%%%%%%%%%%%%%%%%%%%%%%%%%%%%%%%%%%%%%%%%%%%%%%%%%
% Espacio vertical en TOC
%\makeatletter
%\pretocmd{\part}{\addtocontents{toc}{\protect\addvspace{10\p@}}}{}{}
%\pretocmd{\chapter}{\addtocontents{toc}{\protect\addvspace{2\p@}}}{}{}
%\makeatother
%%%%%%%%%%%%%%%%%%%%%%%%%%%%%%%%%%%%%%%%%%%%%%%%%%%%%%%%%%%%%%%%%%%%%%
%Formato de letra
\usepackage{lmodern}
\renewcommand*\familydefault{\sfdefault} %% Only if the base font of the document 
%%%%%%%%%%%%%%%%%%%%%%%%%%%%%%%%%%%%%%%%%%%%%%%%%%%%%%%%%%%%%%%%%%%%%%%%%%%%%%%%%%%%%%%%%%%%
\renewcommand{\spanishtablename}{Tabla}% escribe Tabla
%\renewcommand*\listtablename{Índice de tablas}
%\renewcommand{\contentsname}{Contidos do PFG}
%%%%%%%%%%%%%%%%%%%%%%%%%%%%%%%%%%%%%%%%%%%%%%%%%%%%%%%%%%%%%%%%%%%%%%%%%%%%%%%%
\setcounter{secnumdepth}{5}% Profundidad del índice de contenidos
%%%%%%%%%%%%%%%%%%%%%%%%%%%%%%%%%%%%%%%%%%%%%%%%%%%%%%%%%%%%%%%%%%%%%%%%%%%%%%%%%%%%%%%%%%%%%%%%%%%%%%%%%%%%%%%%%%%%%%%%%%%%%%%%%%%%%%%%%%%%%%%%%%%%%%%%%%%%%%%%%%%%%%%%%%%%%
\numberwithin{equation}{subsection}
\usepackage{chngcntr}
\counterwithin{table}{subsection}% Numera las tablas por sucsecciones
\counterwithin{figure}{subsection}% Numera las figuras por sucsecciones

\DeclareCaptionLabelSeparator{guion}{\ --\ }
\captionsetup[figure]{labelsep=guion}% establece un guión como separador en el pie de figura
\captionsetup[table]{labelsep=guion}% establece un guión como separador en el pie de tabla
%%%%%%%%%%%%%%%%%%%%%%%%%%%%%%%%%%%%%%%%%%%%%%%%%%%%%%%%%%%%%%%%%%%%%%%%%%%%%%%%%%%%%%%%%%%%%%%%%%%%%%%%%%%%%%%%%%%%%%%%%%%%%%%%%%%
\usepackage{float}
\newfloat{Plano}{p}{pln}%[chapter]
\captionsetup[Plano]{labelformat=empty,labelsep=none,position=below}
%%%%%%%%%%%%%%%%%%%%%%%%%%%%%%%%%%%%%%%%%%%%%%%%%%%%%%%%%%%%%%%%%%%%%%%%%%%%%%%%%%%%%%%%%%%%%%%%%%%%%%%%%%%%%%%%%%%%%%%%%%%%%%%%%%%%%%%%%%%%%
\usepackage{float}
\newfloat{Circuito}{c}{cir}%[chapter]
\captionsetup[Circuito]{labelformat=empty,labelsep=none}
%%%%%%%%%%%%%%%%%%%%%%%%%%%%%%%%%%%%%%%%%%%%%%%%%%%%%%%%%%%%%%%%%%%%%%%%%%%%%%%%%%%%%%%%%%%%%%%%%%%%%%%%%%%%%%%%%%%%%%%%%%%%%%%%%%%%%%%%%%%%%
\usepackage[subfigure]{tocloft}% Permite cambiar el ancho de la numeración de listas de figuras y tablas
\addtolength{\cfttabnumwidth}{17pt}
\addtolength{\cftfignumwidth}{17pt}
\makeatletter
\let\l@lstlisting\l@figure
\makeatother
%%%%%%%%%%%%%%%%%%%%%%%%%%%%%%%%%%%%%%%%%%%%%%%%%%%%%%%%%%%%%%%%%%%%%%%%%%%%%%%%%%%%%%%%%%%%%%%%%%%%%%%%%%%%%%%%%%%%%%%%%%%%%%%%%%%%%%%%%%
\patchcmd{\chapter}{plain}{fancy}{}{}% Permite que el formato de la primera página sea como los demás
%%%%%%%%%%%%%%%%%%%%%%%%%%%%%%%%%%%%%%%%%%%%%%%%%%%%%%%%%%%%%%%%%%%%%%%%%%%%%%%%%%%%%%%%%%%%%%%%%%%%%%%%%%%%%%%%%%%%%%%%%%%%%%%%%%%%%%%%%%
\AtBeginDocument{\addtocontents{toc}{\protect\thispagestyle{fancy}}} % Permite que el formato de la página de tableofcontents sea como los demás
%%%%%%%%%%%%%%%%%%%%%%%%%%%%%%%%%%%%%%%%%%%%%%%%%%%%%%%%%%%%%%%%%%%%%%%%%%%%%%%%%%%%%%%%%%%%%%%%%%%%%%%%%%%%%%%
% Uso de las cabeceras fancy
\fancyhf{}
\fancyfoot[R]{\small \thepage}
\fancyfoot[C]{\small \raisebox{0pt}{\documento}}
\fancyhead[L]{\small \titulouno \ \\ \alumno}
% Introducir la especialidad-----------------------------------------------------------------------------------------------------
\fancyhead[C]{}% Escribir ELECTRICIDAD o ELECTRÓNICA                                                         -
% Introducir el número del PFG---------------------------------------------------------------------------------------------------
\fancyhead[R]{}% E es la especialidad 1=Electrónica 2=Electricidad   XXX es el número del proyecto     -
% Introducir la convocatoria del PFG---------------------------------------------------------------------------------------------
\fancyfoot[L]{}% Por ejemplo JUNIO 2014

\renewcommand{\headrulewidth}{0.5pt}
\renewcommand{\footrulewidth}{0.5pt}
%%%%%%%%%%%%%%%%%%%%%%%%%%%%%%%%%%%%%%%%%%%%%%%%%%%%%%%%%%%%%%%%%%%%%%%%%%%%%%%%%%%%%%%%%%%%%%%%%%%%%%%%%%%%%%%%%%%%%%%%%%%%%%%
% Paquete hyperref
\usepackage[colorlinks=true,linkcolor=red,citecolor=red,urlcolor=red]{hyperref}
%%%%%%%%%%%%%%%%%%%%%%%%%%%%%%%%%%%%%%%%%%%%%%%%%%%%%%%%%%%%%%%%%%%%%%%%%%%%%%%%%%%%%%%%%%%%%%%%%%%%
% Se cambio el nombre del caption para los códigos
\renewcommand\lstlistingname{Código}
%%%%%%%%%%%%%%%%%%%%%%%%%%%%%%%%%%%%%%%%%%%%%%%%%%%%%%%%%%%%%%%%%%%%%%%%%%%%%%%%%%%%%%%%%%%%%%%%%%%%%%%%%%%%%%%%%%%%%%%%%%%%%%%%%%
%  Permite meter figuras partidas en páginas
\newcommand\Image[3][]{
  \tabular[b]{@{}c@{}}\includegraphics[#1]{#2}\\
    {\small #3}
  \endtabular}
%%%%%%%%%%%%%%%%%%%%%%%%%%%%%%%%%%%%%%%%%%%%%%%%%%%%%%%%%%%%%%%%%%%%%%%%%%%%%%%%%%%%%%%%%%%%%%%%%%%%%%%%%%%%%%%%%%%%%%%%%%%%%%%%%%



