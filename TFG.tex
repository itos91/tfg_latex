\documentclass[11pt,twoside]{book}
\usepackage[utf8]{inputenc}
\usepackage[spanish]{babel}
%%%%%%%%%%%%%%%%%%%%%%%%%%%%%%%%%%%%%%%%%%%%%%%%%%%%%%%%%%%%%%%%%%%%%%%%%%%%%%%%%%%%%%%%%%%%%%%%%%%%%%%
\usepackage{afterpage}% Necesario para introdicur páginas A3
\usepackage{amsmath,amsthm,amstext,amssymb}
\usepackage{capt-of}
\usepackage{colortbl}
\usepackage{graphicx}% Permite la introducción de figuras
\usepackage{array,fancyhdr,graphicx,subfigure,titlesec,titletoc,xcolor}
\usepackage{emptypage}% evita la numeración de las páginas en blanco
\usepackage{etoolbox}
\usepackage{listings} % premite la introducción de códigos vhdl
\usepackage{minted}
\usepackage[paper=A4,pagesize]{typearea}%necesario para introducir páginas A3
\usepackage{lscape}% Necesario para páginas apaisadas
\usepackage{pdfpages}% Permite introducir documentos pdf
\usepackage[font=small,bf]{caption}% Formato del caption
\usepackage{rotating}% Rotaciones
\usepackage{setspace,subfigure}
\usepackage{tocstyle}
\usepackage{codigomatlab}
\usepackage{titlesec}
\usepackage{color}
\usepackage{tikz}
\usetikzlibrary{calc}
\usepackage{pstricks}
\usepackage{pst-node}
\usepackage{pst-blur}
\usepackage{amsmath,amssymb}
\usepackage{booktabs}
\usepackage{xcolor}
\usepackage{colortbl}
\usepackage{rotating}
\usepackage{multirow}
\usepackage{bigstrut}
\usepackage{eurosym}
%\usepackage{inputenc}
%%%%%%%%%%%%%%%%%%%%%%%%%%%%%%%%%%%%%%%%%%%%%%%%%%%%%%%%%%%%%%%%%%%%%%%%%%%%%%%%%%%%%%%%%%%
\newcommand{\documento}{ÍÍNDICE XERAL}
%%%%%%%%%%%%%%%%%%%%%%%%%%%%%%%%%%%%%%%%%%%%%%%%%%%%%%%%%%%%%%%%%%%%%%%%%%%%%%%%%%%%%%%%%%%
% Incluye la bibliografía como sección
\makeatletter
\renewenvironment{thebibliography}[1]
     {\chapter{\bibname}% esta línea cambia la bibliografía a la categoría sección
      \@mkboth{\MakeUppercase\bibname}{\MakeUppercase\bibname}
      \list{\@biblabel{\@arabic\c@enumiv}}%
           {\settowidth\labelwidth{\@biblabel{#1}}
            \leftmargin\labelwidth
            \advance\leftmargin\labelsep
            \@openbib@code
            \usecounter{enumiv}
            \let\p@enumiv\@empty
            \renewcommand\theenumiv{\@arabic\c@enumiv}}
      \sloppy
      \clubpenalty4000
      \@clubpenalty \clubpenalty
      \widowpenalty4000%
      \sfcode`\.\@m}
     {\def\@noitemerr
       {\@latex@warning{Empty `thebibliography' environment}}
      \endlist}
\makeatother
%%%%%%%%%%%%%%%%%%%%%%%%%%%%%%%%%%%%%%%%%%%%%%%%%%%%%%%%%%%%%%%%%%%%%%%%%%%%%%%%%%%%%%%%%%%%%%%%%%%%%%%%%%%%%%%%%%%%%%%%%%%%%%%%%%%%%%%%%%%%%%%%%%%%%%%%%%%%%%%%%%%%%
%FORMATO DE LA HOJA
\renewcommand{\baselinestretch}{1.25}
\headsep 8mm        \topmargin -1.5cm     \textheight 24.5cm     \textwidth 16cm     \oddsidemargin 0.5cm     \evensidemargin -0.1cm
 \footnotesep=20pt               \footskip=38pt
%%%%%%%%%%%%%%%%%%%%%%%%%%%%%%%%%%%%%%%%%%%%%%%%%%%%%%%%%%%%%%%%%%%%%%%%%%%%%%%%%%%%%%%%%%%%%%%%%%%%%%%%%%%%%%%%%%%%%%%%%%%%%%%%%%%%%%%%%%%%%%%%%%%%%%%%%%%%%%%%%%%%%
% Formato del título de cada parte
\titleformat{\part}[display]
  {\normalfont\huge\bfseries}
	{}{0pt}{\centering}
	
\titlespacing*{\part}{0pt}{0pt}{20pt}
\titleclass{\part}{straight}
%%%%%%%%%%%%%%%%%%%%%%%%%%%%%%%%%%%%%%%%%%%%%%%%%%%%%%%%%%%%%%%%%%%%%%%%
%Formato del título de cada capítulo
\titleformat{\chapter}[hang] 
{\normalfont\huge\bfseries}{\thechapter}{1em}{} 
%%%%%%%%%%%%%%%%%%%%%%%%%%%%%%%%%%%%%%%%%%%%%%%%%%%%%%%%%%%%%%%%%%%%%%%%
% Espacio vertical en TOC
%\makeatletter
%\pretocmd{\part}{\addtocontents{toc}{\protect\addvspace{10\p@}}}{}{}
%\pretocmd{\chapter}{\addtocontents{toc}{\protect\addvspace{2\p@}}}{}{}
%\makeatother
%%%%%%%%%%%%%%%%%%%%%%%%%%%%%%%%%%%%%%%%%%%%%%%%%%%%%%%%%%%%%%%%%%%%%%
%Formato de letra
\usepackage{lmodern}
\renewcommand*\familydefault{\sfdefault} %% Only if the base font of the document 
%%%%%%%%%%%%%%%%%%%%%%%%%%%%%%%%%%%%%%%%%%%%%%%%%%%%%%%%%%%%%%%%%%%%%%%%%%%%%%%%%%%%%%%%%%%%
\renewcommand{\spanishtablename}{Tabla}% escribe Tabla
%\renewcommand*\listtablename{Índice de tablas}
%\renewcommand{\contentsname}{Contidos do PFG}
%%%%%%%%%%%%%%%%%%%%%%%%%%%%%%%%%%%%%%%%%%%%%%%%%%%%%%%%%%%%%%%%%%%%%%%%%%%%%%%%
\setcounter{secnumdepth}{5}% Profundidad del índice de contenidos
%%%%%%%%%%%%%%%%%%%%%%%%%%%%%%%%%%%%%%%%%%%%%%%%%%%%%%%%%%%%%%%%%%%%%%%%%%%%%%%%%%%%%%%%%%%%%%%%%%%%%%%%%%%%%%%%%%%%%%%%%%%%%%%%%%%%%%%%%%%%%%%%%%%%%%%%%%%%%%%%%%%%%%%%%%%%%
\numberwithin{equation}{subsection}
\usepackage{chngcntr}
\counterwithin{table}{subsection}% Numera las tablas por sucsecciones
\counterwithin{figure}{subsection}% Numera las figuras por sucsecciones

\DeclareCaptionLabelSeparator{guion}{\ --\ }
\captionsetup[figure]{labelsep=guion}% establece un guión como separador en el pie de figura
\captionsetup[table]{labelsep=guion}% establece un guión como separador en el pie de tabla
%%%%%%%%%%%%%%%%%%%%%%%%%%%%%%%%%%%%%%%%%%%%%%%%%%%%%%%%%%%%%%%%%%%%%%%%%%%%%%%%%%%%%%%%%%%%%%%%%%%%%%%%%%%%%%%%%%%%%%%%%%%%%%%%%%%
\usepackage{float}
\newfloat{Plano}{p}{pln}%[chapter]
\captionsetup[Plano]{labelformat=empty,labelsep=none,position=below}
%%%%%%%%%%%%%%%%%%%%%%%%%%%%%%%%%%%%%%%%%%%%%%%%%%%%%%%%%%%%%%%%%%%%%%%%%%%%%%%%%%%%%%%%%%%%%%%%%%%%%%%%%%%%%%%%%%%%%%%%%%%%%%%%%%%%%%%%%%%%%
\usepackage{float}
\newfloat{Circuito}{c}{cir}%[chapter]
\captionsetup[Circuito]{labelformat=empty,labelsep=none}
%%%%%%%%%%%%%%%%%%%%%%%%%%%%%%%%%%%%%%%%%%%%%%%%%%%%%%%%%%%%%%%%%%%%%%%%%%%%%%%%%%%%%%%%%%%%%%%%%%%%%%%%%%%%%%%%%%%%%%%%%%%%%%%%%%%%%%%%%%%%%
\usepackage[subfigure]{tocloft}% Permite cambiar el ancho de la numeración de listas de figuras y tablas
\addtolength{\cfttabnumwidth}{17pt}
\addtolength{\cftfignumwidth}{17pt}
\makeatletter
\let\l@lstlisting\l@figure
\makeatother
%%%%%%%%%%%%%%%%%%%%%%%%%%%%%%%%%%%%%%%%%%%%%%%%%%%%%%%%%%%%%%%%%%%%%%%%%%%%%%%%%%%%%%%%%%%%%%%%%%%%%%%%%%%%%%%%%%%%%%%%%%%%%%%%%%%%%%%%%%
\patchcmd{\chapter}{plain}{fancy}{}{}% Permite que el formato de la primera página sea como los demás
%%%%%%%%%%%%%%%%%%%%%%%%%%%%%%%%%%%%%%%%%%%%%%%%%%%%%%%%%%%%%%%%%%%%%%%%%%%%%%%%%%%%%%%%%%%%%%%%%%%%%%%%%%%%%%%%%%%%%%%%%%%%%%%%%%%%%%%%%%
\AtBeginDocument{\addtocontents{toc}{\protect\thispagestyle{fancy}}} % Permite que el formato de la página de tableofcontents sea como los demás
%%%%%%%%%%%%%%%%%%%%%%%%%%%%%%%%%%%%%%%%%%%%%%%%%%%%%%%%%%%%%%%%%%%%%%%%%%%%%%%%%%%%%%%%%%%%%%%%%%%%%%%%%%%%%%%
% Uso de las cabeceras fancy
\fancyhf{}
\fancyfoot[R]{\small \thepage}
\fancyfoot[C]{\small \raisebox{0pt}{\documento}}
\fancyhead[L]{\small \titulouno \ \\ \alumno}
% Introducir la especialidad-----------------------------------------------------------------------------------------------------
\fancyhead[C]{}% Escribir ELECTRICIDAD o ELECTRÓNICA                                                         -
% Introducir el número del PFG---------------------------------------------------------------------------------------------------
\fancyhead[R]{}% E es la especialidad 1=Electrónica 2=Electricidad   XXX es el número del proyecto     -
% Introducir la convocatoria del PFG---------------------------------------------------------------------------------------------
\fancyfoot[L]{}% Por ejemplo JUNIO 2014

\renewcommand{\headrulewidth}{0.5pt}
\renewcommand{\footrulewidth}{0.5pt}
%%%%%%%%%%%%%%%%%%%%%%%%%%%%%%%%%%%%%%%%%%%%%%%%%%%%%%%%%%%%%%%%%%%%%%%%%%%%%%%%%%%%%%%%%%%%%%%%%%%%%%%%%%%%%%%%%%%%%%%%%%%%%%%
% Paquete hyperref
\usepackage[colorlinks=true,linkcolor=red,citecolor=red,urlcolor=red]{hyperref}
%%%%%%%%%%%%%%%%%%%%%%%%%%%%%%%%%%%%%%%%%%%%%%%%%%%%%%%%%%%%%%%%%%%%%%%%%%%%%%%%%%%%%%%%%%%%%%%%%%%%
% Se cambio el nombre del caption para los códigos
\renewcommand\lstlistingname{Código}
%%%%%%%%%%%%%%%%%%%%%%%%%%%%%%%%%%%%%%%%%%%%%%%%%%%%%%%%%%%%%%%%%%%%%%%%%%%%%%%%%%%%%%%%%%%%%%%%%%%%%%%%%%%%%%%%%%%%%%%%%%%%%%%%%%
%  Permite meter figuras partidas en páginas
\newcommand\Image[3][]{
  \tabular[b]{@{}c@{}}\includegraphics[#1]{#2}\\
    {\small #3}
  \endtabular}
%%%%%%%%%%%%%%%%%%%%%%%%%%%%%%%%%%%%%%%%%%%%%%%%%%%%%%%%%%%%%%%%%%%%%%%%%%%%%%%%%%%%%%%%%%%%%%%%%%%%%%%%%%%%%%%%%%%%%%%%%%%%%%%%%%




%%%%%%%%%%%%%%%%%%%%%%%%%%%%%%%%%%%%%%%%%%%%%%%%%%%%%%%%%%%%%%%%%%%%%%%%%%%%%%%%%%%%%%%%%%%%%%%%%%%%%%%%%
%%%%%%%%%%%%%%%%%%%%%%%%%%%%%%%%%%%%%%%%%%%%%%%%%%%%%%%%%%%%%%%%%%%%%%%%%%%%%%%%%%%%%%%%%%%%%%%%%%%%%%%%%
%--------------------------------------------------------------------------------------------------------
%             Datos del ALUMNO
%--------------------------------------------------------------------------------------------------------
\newcommand{\alumno}{
% 1 Se debe introducir el NOMBRE y APELLIDOS del alumno (EN MAY�SCULAS)
Ávaro Fernádez Quesada
}
\newcommand{\especialidad}{
% 2 Introducir LA ESPECIALIDAD: ELECTR�NICA o ELECTRICA
Grao en Enxeñeríaren Electrónica Industrial e Automática
}
\newcommand{\grado}{
% 3 Introducir el grado: Electr�nica Industrial y Autom�tica o El�ctrica
Grao en Enxeñería en Electrónica Industrial e Automática
}
\newcommand{\materia}{
% 4 Introducir el nombre de la asignatura a la que se asocia el PFG
ASIGNATURA
}
%%%%%%%%%%%%%%%%%%%%%%%%%%%%%%%%%%%%%%%%%%%%%%%%%%%%%%%%%%%%%%%%%%%%%%%%%%%%%%%%%%%%%%%%%%%%%%%%%%%%%%%%%

%%%%%%%%%%%%%%%%%%%%%%%%%%%%%%%%%%%%%%%%%%%%%%%%%%%%%%%%%%%%%%%%%%%%%%%%%%%%%%%%%%%%%%%%%%%%%%%%%%%%%%%%%
%--------------------------------------------------------------------------------------------------------
%              Datos del PFG
%--------------------------------------------------------------------------------------------------------
% 5 Introducir EL T�TULO COMPLETO DEL PFG (EN MAY�SCULAS)
% El t�tulo suele ser largo y generalmente no cabe en una l�nea. La plantilla est� preparada para escribir el t�tulo en una, dos o tres l�neas. 
%------------------- BLOQUE: PRIMERA L�NEA DE T�TULO
\newcommand{\titulouno}{
Terminal de operador inalámbrico para preparación de pedidos dun almacén
}
%------------------- BLOQUE: S� HAY SEGUNDA L�NEA DE T�TULO
%-------------------
% Si existe segunda l�nea del t�tulo, activar (BORRAR %) este bloque y comentar (ESCRIBIR % AL PRINCIPIO DE CADA L�NEA) el siguiente. En la tercera l�nea de esta bloque se escribir� esa parte del t�tulo.
\newcommand{\titulodos}{
\colorbox{lightgray}{\large\bf 
 SEGUNDA L�NEA  DEL T�TULO
}}
%------------------- BLOQUE: NO HAY SEGUNDA L�NEA DE T�TULO
%-------------------
% Si no existe segunda l�nea del t�tulo, activar la siguiente l�nea y comentar el bloque anterior.
%\newcommand{\titulodos}{}
%
%------------------- BLOQUE: S� HAY TERCERA L�NEA DE T�TULO
%-------------------
% Si existe tercera l�nea del t�tulo, activar (BORRAR %) este bloque y comentar (ESCRIBIR % AL PRINCIPIO DE CADA L�NEA) el siguiente. 
% En la tercera l�nea de esta bloque se escribir� esa parte del t�tulo.
\newcommand{\titulotres}{
\colorbox{lightgray}{\large\bf 
TERCERA L�NEA  DEL T�TULO
}}
%------------------- BLOQUE: NO HAY TERCERA L�NEA DE T�TULO
%-------------------
% Si no existe tercera l�nea del t�tulo, activar la siguiente l�nea y comentar el bloque anterior.
%\newcommand{\titulotres}{}

%--------------------------------------------------------------------------------------------------
\newcommand{\numero}{
% 6 Introducir EL N�MERO DEL PFG (EN MAY�SCULAS)
770G0XAXX
}
%--------------------------------------------------------------------------------------------------
\newcommand{\convocatoria}{
% 7 Introducir LA CONVOCATORIA (EN MAY�SCULAS)
MES
}
%--------------------------------------------------------------------------------------------------
\newcommand{\anho}{
% 8 Introducir EL A�O
20XX
}
%--------------------------------------------------------------------------------------------------
\newcommand{\tutoruno}{
% 9 Introducir datos del tutor NOMBRE y APELLIDOS (EN MAY�SCULAS)
José Luis Camaño Portela
}
%------------------- BLOQUE: S� HAY SEGUNDO TUTOR
% Si hay segundo tutor, activar (BORRAR %) este bloque y comentar (ESCRIBIR % AL PRINCIPIO DE CADA L�NEA) el siguiente. En la tercera l�nea de esta bloque se escribir� esa parte del t�tulo.
\newcommand{\tutordos}{
\colorbox{lightgray}{\large\bf 
NOMBRE APELLIDO1 APELLIDO2
}}
%------------------- BLOQUE: NO HAY SEGUNDO TUTOR
%-------------------
% Si no hay segundo tutor, activar la siguiente l�nea y comentar el bloque anterior.
%\newcommand{\tutordos}{}
%%%%%%%%%%%%%%%%%%%%%%%%%%%%%%%%%%%%%%%%%%%%%%%%%%%%%%%%%%%%%%%%%%%%%%%%%%%%%%%%%%%%%%%%%%%%%%%%%%%%%%%%%
%%%%%%%%%%%%%%%%%%%%%%%%%%%%%%%%%%%%%%%%%%%%%%%%%%%%%%%%%%%%%%%%%%%%%%%%%%%%%%%%%%%%%%%%%%%%%%%%%%%%%%%%%







\include{Listado_arduino}% Opciones: vhdl, arduino
%%%%%%%%%%%%%%%%%%%%%%%%%%%%%%%%%%%%%%%%%%%%%%%%%%%%%%%%%%%%%%%%%%%%%%%%%%%%%%%%%%%%%%%%%%%%%%%%%%%%%%%%%%%%%%%%%
\begin{document}
% Se incluye la portada del TFG
\pagestyle{empty}

\begin{tikzpicture}[remember picture, overlay]
  \draw[line width = 0.5pt] ($(current page.north west) + (86pt,-43.5pt)$) rectangle ($(current page.south east) + (-46.5pt,43.5pt)$);
\end{tikzpicture}

\begin{center}
\begin{figure}[htbp]
\begin{center}
\includegraphics[angle=0, height=3.8cm]{images/EEILogo.png}
\end{center}
\end{figure}
\ \\
\begin{large}
\begin{center}
\color{blue}\textbf{Escola de Enxeñería Industrial}
\end{center}
\end{large}
\ \\
\ \\
\begin{large}
\begin{center}
\textbf{TRABALLO FIN DE GRAO}
\end{center}
\end{large}
\ \\
\ \\
\begin{large}
\begin{center}
{\titulouno}
\end{center}
\end{large}
\ \\
\ \\
\begin{normalsize}
\begin{center}
\textbf{\grado}
\end{center}
\end{normalsize}
\ \\
\ \\
\ \\
\ \\
\begin{normalsize}
\begin{center}
\textbf{Alumno:  \qquad \qquad \alumno}
\end{center}
\end{normalsize}
\ \\
\begin{normalsize}
\begin{center}
\textbf{Directores: \qquad \qquad \tutoruno}
\end{center}
\end{normalsize}
\ \\
\ \\
\ \\
\ \\

\begin{center}
\begin{figure}[htbp]
\begin{center}
\includegraphics[angle=0, height=0.8cm]{images/UVIGOLogo.png}
\end{center}
\end{figure}
\end{center}

\end{center}


\cleardoublepage
%%%%%%%%%%%%%%%%%%%%%%%%%%%%%%%%%%
% Página de resumen del proyecto %
%%%%%%%%%%%%%%%%%%%%%%%%%%%%%%%%%%

\thispagestyle{empty}

\bigskip
\bigskip

\large{
\textbf{Resumo:}}

\bigskip
\bigskip


\begin{center}
\textbf{\titulouno}
\end{center}

\bigskip
\bigskip
%\bigskip
\large{
\textbf{Alumno:}}\alumno

%\medskip
\large{
\textbf{Director:}} \tutoruno

%\vfill

%\begin{minipage}{\textwidth}
%\textbf{Dpto. de:}
%


%\medskip
%
%\textbf{Titulación:} Ingeniería de Telecomunicación
%
%\medskip
\bigskip
\bigskip


O presente proxecto lévase a cabo a implementación de un sistema de xestión de preparación de pedidos de varios productos nun almacén.

Desarrollouse un programa elaborado en C++ para unha placa Arduino para que sirva como terminal de operador para un carretilleiro de un almacén e que desde o cal podan xestionar encargos comunicándose por WiFi a un servidor.

O terminal de operador está composto por unha pantalla LCD para visualizar en todo momento o encargo e por un teclado matricial para indicar que o producto depositouse nunha caixa.
A comunicación por WiFi é levada a cabo polo módulo ESP8266 mediante servicios REST a un servidor que dispón dunha base de datos relacional.

O entorno de desenvolvemento dos servicios REST fixéronse con un framework de python chamado API RESTful Django e o programa para Arduino, por Atom.
 

\bigskip
\bigskip

\textbf{Palabras clave:} IoT, Picking, Almacén, Arduino, Servicios RESTful, Python.

%\begin{center} Vigo, \today\end{center}
%\end{minipage}

%Página en blanco
\newpage{\pagestyle{empty}\cleardoublepage}
\cleardoublepage
%%%%%%%%%%%%%%%%%%%%%%%%%%%%%%%%%%%%%%%%%%%%%%%%%%%%%%%%%%%%%%%%%%%%%%%%%%%%%%%%%%%%%%%%%%%%%%%%%%%%%%%%%%%%%%%%%%
% DOCUMENTO ÍNDICE XERAL
\pagestyle{empty}

\begin{tikzpicture}[remember picture, overlay]
  \draw[line width = 0.5pt] ($(current page.north west) + (-20pt,-800.5pt)$) rectangle ($(current page.south east) + (-133.5pt,-722.5pt)$);
\end{tikzpicture}

\renewcommand{\document}{ÍNDICE XERAL}

\begin{center}
\begin{figure}[htbp]
\begin{center}
\includegraphics[angle=0, height=3.8cm]{images/EEILogo.png}
\end{center}
\end{figure}
\ \\
\begin{large}
\begin{center}
\color{blue}\textbf{Escola de Enxeñería Industrial}
\end{center}
\end{large}
\ \\
\ \\
\begin{large}
\begin{center}
\textbf{TRABALLO FIN DE GRAO}
\end{center}
\end{large}
\ \\
\ \\
\begin{large}
\begin{center}
{\titulouno}
\end{center}
\end{large}
\ \\
\ \\
\begin{normalsize}
\begin{center}
\textbf{\grado}
\end{center}
\end{normalsize}
\ \\
\ \\
\ \\
\ \\
\begin{normalsize}
\begin{center}
\textbf{Documento}
\end{center}
\end{normalsize}
\ \\
\begin{normalsize}
\begin{center}
\part{\bf{ÍNDICE XERAL}}\thispagestyle{empty}
\end{center}
\end{normalsize}
\ \\
\ \\
\ \\
\ \\

\begin{center}
\begin{figure}[htbp]
\begin{center}
\includegraphics[angle=0, height=0.8cm]{images/UVIGOLogo.png}
\end{center}
\end{figure}
\end{center}

\end{center}



%%%%%%%%%%%%%%%%%%%%%%%%%%%%%%%%%%%%%%%%%%%%%%%%%%%%%%%%%%%%%%%%%%%%%%%%%%%%%%%%%%%%%%%%%%%%%%%%%%%%%%%%%%%%%%%%%%
\cleardoublepage



% Página que contiene el Índice de contenidos del TFG
\renewcommand{\contentsname}{Contido do TFG}
\addcontentsline{toc}{section}{Contidos do TFG}
%\setcounter{tocdepth}{3}
{\hypersetup{hidelinks}\tableofcontents}
\addtocontents{toc}{\protect\thispagestyle{fancy}}

% Página que contiene el Índice de listas de figuras
\cleardoublepage
\phantomsection
\renewcommand*\listfigurename{Índice de figuras}
\addcontentsline{toc}{section}{\listfigurename}
{\hypersetup{hidelinks}\listoffigures}
\addtocontents{lof}{\protect\thispagestyle{fancy}}

% Página que contiene el índice de listas de tablas
\cleardoublepage
\phantomsection
\renewcommand*\listtablename{Índice de tablas}
\addcontentsline{toc}{section}{\listtablename}
{\hypersetup{hidelinks}\listoftables}
\addtocontents{lot}{\protect\thispagestyle{fancy}}

\cleardoublepage

%%%%%%%%%%%%%%%%%%%%%%%%%%%%%%%%%%%%%%%%%%%%%%%%%%%%%%%%%%%%%%%%%%%%%%%%%%%%%%%%%%%%%%%%%%%%%%%%%%%%%%%%%%%%%
%%%%%%%%%%%%%%%%%%%%%%%%%%%%%%%%%%%%%%%%%%%%%%%%%%%%%%%%%%%%%%%%%%%%%%%%%%%%%%%%%%%%%%%%%%%%%%%%%%%%%%%%%%%%%
% DOCUMENTO MEMORIA
\pagestyle{empty}

\begin{tikzpicture}[remember picture, overlay]
  \draw[line width = 0.5pt] ($(current page.north west) + (-20pt,-800.5pt)$) rectangle ($(current page.south east) + (-133.5pt,-722.5pt)$);
\end{tikzpicture}

\renewcommand{\documento}{MEMORIA}

\begin{center}
\begin{figure}[htbp]
\begin{center}
\includegraphics[angle=0, height=3.8cm]{images/EEILogo.png}
\end{center}
\end{figure}
\ \\
\begin{large}
\begin{center}
\color{blue}\textbf{Escola de Enxeñería Industrial}
\end{center}
\end{large}
\ \\
\ \\
\begin{large}
\begin{center}
\textbf{TRABALLO FIN DE GRAO}
\end{center}
\end{large}
\ \\
\ \\
\begin{large}
\begin{center}
{\titulouno}
\end{center}
\end{large}
\ \\
\ \\
\begin{normalsize}
\begin{center}
\textbf{\grado}
\end{center}
\end{normalsize}
\ \\
\ \\
\ \\
\ \\
\begin{normalsize}
\begin{center}
\textbf{Documento}
\end{center}
\end{normalsize}
\ \\
\begin{normalsize}
\begin{center}
\part{\bf{MEMORIA}}
\end{center}
\end{normalsize}
\ \\
\ \\
\ \\
\ \\

\begin{center}
\begin{figure}[htbp]
\begin{center}
\includegraphics[angle=0, height=0.8cm]{images/UVIGOLogo.png}
\end{center}
\end{figure}
\end{center}

\end{center}

\cleardoublepage


\pagestyle{fancy}
%%%%%%%%%%%%%%%%%%%%%%%%%%%%%%%%%%%%%%%%%%%%%%%%%%%%%%%%%%%%%%%%%%%%%%%%%%%%%%%%%%%%%%%%%%%%%%%%%%%%%%%%%%%%%%
%%%%%%%%%%%%%%%%%%%%%%%%%%%%%%%%%%%%%%%%%%%%%%%%%%%%%%%%%%%%%%%%%%%%%%%%%%%%%%%%%%%%%%%%%%%%%%%%%%%%%%%%%%%%%

\addcontentsline{toc}{section}{Índice do documento Memoria}
\startcontents[parts]
\begin{center}{\large \bf Índice de MEMORIA}\end{center}

{\hypersetup{hidelinks}\printcontents[parts]{}{-1}{\setcounter{tocdepth}{5}}}

\cleardoublepage%----------------------------------------

\chapter{Introducción}

\section{Estado de arte}

Hoxe en día todo está conectado á rede, non solo persoas conectadas a través de un dispositivo, senón a calqueira "cousa" que teña algunha funcionalidade e que poida ser controlado desde calqueira dispositivos, xa sexa desde un smartphone, un ordenador ou desde un smartwatch. Esto é debido á implementación de IoT (Internet of Things). Pódese definir como \textit{<<unha rede que conecta ``cousa'' identificables de maneira única a Internet. As ``cousas'' teñen capacidades de captura/actuación e de potencial de programación. Mediante a explotación  da identificación e captura única, pódese recoller información mais cambiar o estado da ``cousa'' onde queiras, en calquer momento e por calquer motivo>>} \cite{IoT}

\begin{figure}[H]
	\begin{center}
		\includegraphics[width=10cm]{images/IoT.png}
	\end{center}
	\caption{Internet of Things *http://www.pcworldenespanol.com/}
	\label{fig:IoT}
\end{figure}

\section{Obxeto}

O obxetivo principal deste proxecto vai ser o de realizar a monitorización de un procesado de un pedido ou picking nun almacén para un operario a través dun terminal de operador inalámbrico.

\begin{figure}[H]
	\begin{center}
		\includegraphics[width=10cm]{images/esquema_xeral.png}
	\end{center}
	\caption{Esquema xeral do proxecto}
	\label{fig:IoT}
\end{figure}

Debido ós altos costes dos terminales, como por exemplo SYMBOL VC5090, pensouse en realizalo cunha placa Arduino e o módulo WiFi ESP8266 pola súa capacidade de programación e, por suposto, polo seu precio.
Por outro lado, tamén conta cunha interfaz web para o almacén, podendo xestionar tanto os productos coma os pedidos.

\begin{figure}[H]
	\begin{center}
		\includegraphics[width=10cm]{images/symbol_VC5090.jpg}
	\end{center}
	\caption{Terminal de operador SYMBOL VC5090 *http://www.datanet.com.au/}
	\label{fig:IoT}
\end{figure}

Para elo, vaise dividir o documento \textbf{Memoria} en requerimentos hardware, onde se vai detallar os compoñentes hardware usados, e requerimentos software, onde se explicará a parte de programación da parte de \txtbf{Arduino} e a parte de \textbf{Django RESTful Service}.

\chapter{Conceptos previos}

\section{Preparación de pedidos ou picking}

No mundo da loxística, realizar picking de un producto consiste en ir a unha estantería ou zona concreta dentro do almacén para recoller os productos requeridos para un pedido. É un proceso importante dentro empresas tanto pequenas coma grades, polo que é importante a súa optimización e mecanización.

\subsection{Formas de preparación de pedido}
Dentro picking, existen infinitas metodoloxías tradicionais de xestionalo, que veñen determinadas pola tipoloxía do almacén, niveis de servicios que da a empresa, etc. 
Pódense clasificar segundo:

\begin{itemize}
    \item Onde o efectuamos
        \begin{itemize}
            \item \textbf{De home a pedido:} Os operarios recorren o almacén e seleccionar o producto. 
            \item \textbf{De producto a home:} O producto é o que se desplaza ata o posto no que está o operario. Ten unha maior inversión económica pero reduce o tempo.
        \end{itemize}
    \item Como extraemos a mercancía
        \begin{itemize}
            \item \textbf{Pedido a pedido: } Os operarios preparan de forma individual os pedidos, é decir, solo preparan un pedido á vez. 
            \item \textbf{Por oleaxe: } Realizan picking de varios pedidos á vez. Merece a pena se hai poucos pedidos e moita repetividade. 
        \end{itemize}
    \item A unidade de extracción
        \begin{itemize}
            \item \textbf{De caixas completas.}
            \item \textbf{De unidades soltas.}
        \end{itemize}
\end{itemize}

En concreto neste proxecto, a metodoloxía vai ser picking de home a producto, de pedido a pedido e de unidades soltas.

\section{Arduino}
\subsection{Historia}

Arduino nace polo ano 2005 como un proxecto de estudiantes no Instituto de Diseño Interactivo Ivrea de Italia (IDII), onde os alumnos experimentaban con distintos tipos de microcontroladores. A idea era crear unha ferramenta moderna, sencilla, barata e fácil de usar. Foi así como empezaron a desenvolvela baixo a licencia de Open Source, para que todo o mundo poidese contribuir explotando por todo o mundo.

\subsection{Placas}

Hai moitas variedades de placas. Na táboa de abaixo móstranse as características das principais:

\begin{table}[htb]
\begin{center}
\resizebox{16cm}{!} {
\begin{tabular}{|c|m{3cm}|m{3.5cm}|m{2cm}|m{2cm}|m{2cm}|m{2cm}|m{2cm}|m{2cm}|c|c|}
\hline
Nombre & Procesador & Operating/Voltage Input & CPU Speed & Analogic In/Out & Digital IO/PWM & EEPROM & SRAM & FLASH & USB & UART \\
\hline
101 & Intel Curie & 3.3V/7-12V & 32MHz & 6/0 & 14/4 & - & 24 & 196 & Regular & - \\ 
\hline
Gemma & ATtiny85 & 3.3 V / 4-16 V & 8 MHz & 1/0 & 3/2 & 0.5 & 0.5 & 8 & Micro & 0 \\
\hline
LilyPad & ATmega168V \newline ATmega328P & 2.7-5.5 V/2.7-5.5 V & 8MHz & 6/0 & 14/6 & 0.512 & 1 & 16 & - & - \\
\hline
LilyPad SimpleSnap & ATmega328P & 2.7-5.5 V/2.7-5.5 V & 8 MHz & 4/0 & 9/4 & 1 & 2 & 32 & - & - \\
\hline
LilyPad USB & ATmega32U4 & 3.3 V/3.8-5 V & 8 MHz & 4/0 & 9/4 & 1 & 2.5 & 32 & Micro & - \\
\hline
Mega 2560 & ATmega2560 & 5V/7-12 V & 16 MHz & 16/0 & 54/15 & 4 & 8 & 256 & Regular & 4 \\
\hline
Micro & ATmega32U4 & 5 V/7-12 V & 16 MHz & 12/0 & 20/7 & 1 & 2.5 & 32 & Micro & 1 \\
\hline
MKR1000 & SAMD21 Cortex-M0+ & 3.3 V/ 5V  & 48MHz  & 7/1 & 8/4 & - & 32 & 256 & Micro & 1 \\
\hline
Pro & ATmega168 ATmega328P & 3.3 V/3.35-12 V \newline  5 V/5-12 V & 8 MHz  \newline  16 MHz & 6/0 & 14/6 & 0.512 \newline 1 & 1   2 & 16  32 & - & 1 \\
\hline
Pro Mini & ATmega328P & 3.3 V / 3.35-12 V \newline  5 V / 5-12 V & 8 MHz \newline  16 MHz & 6/0 & 14/6 & 1 & 2 & 32 & - & 1 \\
\hline
Uno & ATmega328P & 5 V / 7-12 V & 16 MHz & 6/0 & 14/6 & 1 & 2 & 32 & Regular & 1 \\
\hline
Zero & ATSAMD21G18 & 3.3 V / 7-12 V & 48 MHz & 6/1 & 14/10 & - & 32 & 256 & 2 Micro & 2 \\
\hline
Due & ATSAM3X8E & 3.3 V / 7-12 V & 84 MHz & 12/2 & 54/12 & - & 96 & 512 & 2 Micro & 4\\
\hline
Esplora & ATmega32U4 & 5 V / 7-12 V & 16 MHz & - & - & 1 & 2.5 & 32 & Micro & - \\
\hline
Ethernet & ATmega328P & 5 V / 7-12 V & 16 MHz & 6/0 & 14/4 & 1 & 2 & 32 & Regular & - \\
\hline
Leonardo & ATmega32U4 & 5 V / 7-12 V & 16 MHz & 12/0 & 20/7 & 1 & 2.5 & 32 & Micro & 1 \\
\hline
Mega ADK & ATmega2560 & 5 V / 7-12 V & 16 MHz & 16/0 & 54/15 & 4 & 8 & 256 & Regular & 4 \\
\hline
Mini & ATmega328P & 5 V / 7-9 V & 16 MHz & 8/0 & 14/6 & 1 & 2 & 32 & - & - \\
\hline
Nano & ATmega168 \newline ATmega328P & 5 V / 7-9 V & 16 MHz & 8/0 & 14/6 & 0.512  1 & 1  2 & 16 32 & Mini & 1 \\
\hline
Yún & ATmega32U4 \newline AR9331 Linux & 5 V & 16 MHz \newline 400MHz & 12/0 & 20/7 & 1 & 2.5 \newline  16MB & 32 \newline   64MB & Micro & 1 \\
\hline
Arduino Robot & ATmega32u4 & 5 V & 16 MHz & 6/0 & 20/6 & 1 KB (ATmega32u4)/512 Kbit (I2C) & 2.5 KB (ATmega32u4) & 32 KB (ATmega32u4) of which 4 KB used by bootloader & 1 & 1 \\
\hline
MKRZero & SAMD21 \newline  Cortex-M0+32bit low power \newline ARM MCU & 3.3 V & 48 MHz & 7 (ADC 8/10/12 bit)/1 (DAC 10 bit) & 22/12 & No & 32 KB & 256 KB & 1 & 1 \\
\hline
\end{tabular}
}
\caption{Comparación de placas Arduino}
\label{taboa:comparacionPlacasArduino}
\end{center}
\end{table}

Neste proxecto vaise usar Arduino Mega 2560.

\subsection{Entorno de Programación}

O entorno de desenvolvemento integrado (Integrated Development Environment ou IDE) é un programa informático composto por un conxunto de ferramentas de programación. 

Arduino ten un IDE propio chamado Arduino IDE, pero por comodidade vaise empregar Atom con un plugin chamado \textit{PlatformIO}, que nos vai permitir poder compilar, subir e depurar código na placa.

\begin{figure}[H]
	\begin{center}
		\includegraphics[width=15cm]{images/Atom.png}
	\end{center}
	\caption{Atom}
	\label{fig:Atom}
\end{figure}

Para a creación dun novo proxecto, faise click en New Project, poñéndolle un nome e seleccionando a placa usada, que neste caso vai ser Arduino Mega 2560.

\begin{figure}[H]
	\begin{center}
		\includegraphics[width=8cm]{images/NewProject.png}
	\end{center}
	\caption{Novo Proxecto}
	\label{fig:NewProject}
\end{figure}

\subsection{Sketches}

Os programas de Arduino, tamén chamados Sketch, están compostos por un solo ficheiro con extensión ``.ino'', pero neste caso, o arquivo vai ter extensión ``.cpp'' como no lenguaxe de programación C++.

\subsection{Librerías}

Na nosa aplicación, podemos incorporar  librerías codificadas por outros programadores. Esto facilita á hora de programar e permite a abstracción facendo que o noso programa sexa moito máis fácil de elaborar e entender.

Disponse de infinidade de librerías para facilitar traballo, sendo todas elas Open Source.

Normalmente veñen comprimidas nun archivo ZIP e conteñen:
\begin{itemize}
\item Un archivo .cpp (código C++)
\item Un arquivo de declaracións con extensión .h que contén a declaración dunha ou varias clases.
\item Un directorio con varios exemplos para axudarnos a entendela.
\end{itemize}

Para incluir librerías no noso proxecto, na carpeta chamada ``lib'' importaranse, creando un packete para cada unha.

\begin{figure}[H]
	\begin{center}
		\includegraphics[width=7cm]{images/librerias.png}
	\end{center}
	\caption{Incluir librerías}
	\label{fig:LibreriasAtom}
\end{figure}

\subsection{Serial Monitor}

PlatformIO ten tamén a opción mostrar o que se comunique ca placa Arduino por medio do porto Serial.
Esto vai ser útil á hora do desenvolvemento xa que nos permite imprimir por pantalla datos ou estadísticas.
 Para eso iremos o apartado Serial Monitor e configuraremos o porto no que temos conectada a placa e os baudios, que é a velocidade de retransmision.
 
 \begin{figure}[H]
	\begin{center}
		\includegraphics[width=7cm]{images/serial_monitor.png}
	\end{center}
	\caption{Serial Monitor}
	\label{fig:LibreriasAtom}
\end{figure}

\section{API REST}

API (Application Programming Incerface) é unha colección de funcións e métodos desenvolvidas para dotar dunha capa de abstracción para o manexo dun dispositivo, estructura de datos ou calqueira outra funcionalidade. 
REST (Representatioal State Transfer) é unha arquitectura de desenvolvemento web que usa o estándar HTTP. 
Con API REST vai permitir o uso de unha función ou método que pertence a unha plataforma para o uso.

\subsection{Arquitectura}

\subsubsection{Identificación de recursos con URI's}

Un recurso representa unha sección, arquivo ou contido que queremos obter ou modificar. Para poder identificalo, usaremos URI's, que teñen que cumprir:
\begin{itemize}
    \item Non se debe usar verbos nos nomes de URI.
    \item Deben ser únicas.
    \item Non deben tomar en conta o formato.
    \item Deben ter unha xerarquía lóxica.
    \item Os filtrados de información faranse mediante parámetros HTTP.
    \item Debe usarse a súa forma plural.
\end{itemize}

\subsubsection{Métodos HTTP}

Os principais métodos son:

\begin{itemize}
    \item GET: Consulta e lectura de recursos.
    \item POST: Creación de recursos.
    \item PUT: Edición de recursos.
    \item DELETE: Eliminación de recursos.
    \item PATCH: Edicion de partes de recursos.
\end{itemize}

\section{Django}

Django é un framework Open Source para o desenvolvemento de aplicacións web. Está escrito na lenguaxe \textbf{Python}. Contén un conxunto de compoñentes que van axudar a compoñer aplicacións web facil e rapidamente.

\subsection{Django API RESTful}

É un microframework de Django que vai permitir a creación de un servicio API REST. A súa estructura básase en 3 compoñentes: \textbf{serializadores}, \textbf{vistas} e \textbf{routers}.

\begin{itemize}
    \item Os routers permiten definir as url da API creada dunha meneira sencilla e arbitraria. Permiten definir que método de unha class view se vai executar ó chegar unha petición HTTP en función do método HTTP (GET, POST, PUT, PATCH...).
    \item As views son extensións das class-view de Django, que permiten facilitar o enganche cos routers, serializadores e modelos.
    \item Os serializadores permiten definir como van ser as respostas que devolve o API e como se procesan o contido das peticións que reciben.
\end{itemize}

Ten a seguinte estructura:

\begin{figure}[H]
	\begin{center}
		\includegraphics[width=4cm]{images/estructura_djangoREST.png}
	\end{center}
	\caption{Estructura Django API RESTful}
	\label{fig:EstructuraDjangoAPI}
\end{figure}


\chapter{Compoñentes Hardware}
\section{Arduino Mega 2560}

Arduino Mega 2560 está basado no microcontralador ATmega2560. Ten 54 pins de entrada/saída dixitais (dos cales 15 pódense empregar como saídas PWM), 16 entradas analóxicas, 4 UART (portas de serie de hardware), un oscilador de cristal de 16 MHz, unha conexión USB, unha toma de enerxía, un encabezado ICSP, e un botón de reset. Contén todo o necesario para soportar o desenvolvimento de aplicacións para este microcontrolador; simplemente conéctase a unha computadora con un cable USB ou aliméntase cun adaptador AC-to-DC ou batería para comezar. 

\begin{figure}[H]
	\begin{center}
		\includegraphics[width=7cm]{images/arduino_mega.jpg}
	\end{center}
	\caption{Arduino Mega 2560 *http://arduino.cc}
	\label{fig:ArduinoMega}
\end{figure}

\subsection{Alimentación eléctrica}

Pode ser alimentado a través da conexión USB ou cunha fonte de alimentación externa. A fonte de enerxía selecciónase automaticamente. 

A placa pode funcionar nunha fonte externa de 6 a 20 voltios. Se se fornece con menos de 7V, con todo, o pin de 5V pode fornecer menos de 5V e pode volverse inestable. Se usa máis de 12V, o regulador de voltaxe pode sobrecalentarse e dañala. O rango recomendado é de 7 a 12V.

Os pines de alimentación son os seguintes:

\begin{itemize}
\item \textbf{Vin.} A tensión de entrada á placa cando está a usar unha fonte de enerxía externa (a diferenza de 5 voltios da conexión USB ou outra fonte de alimentación regulada). Pode fornecer a tensión a través deste pin ou, se fornecer a tensión a través do conector de enerxía, accédese a través deste pin.
\item \textbf{5V.} Este pin outorga un regulador de 5V desde o regulador do taboleiro. O taboleiro pode ser subministrado con enerxía desde a toma de alimentación de CC (7-12V), o conector USB (5V) ou a pin VIN do taboleiro (7-12V). A subministración de tensión a través dos pines de 5V ou 3.3V evita o regulador e pode danar a táboa. Non o aconsellamos.
\item \textbf{3V3.} Unha fonte de 3,3 Voltios xerada polo regulador de a bordo. O amperaxe máximo actual é de 50 mA.
\item \textbf{GND.} Pins de terra.
\item \textbf{IOREF.} Este pin proporciona a referencia de tensión coa que funciona o microcontrolador. Un escudo configurado correctamente pode ler a tensión de tecto IOREF e seleccionar a fonte de enerxía adecuada ou habilitar traductores de tensión nas saídas para traballar cos 5V ou 3.3V.
\end{itemize}

\section{Módulos ou periféricos}

\subsection{Pantalla LCD}

Para visualizar os pedidos dos operarios e xestionalos, empregarase unha pantalla LCD da marca Midas que ten unha resolución de 4 liñas x 40 caracteres. 

\begin{figure}[H]
	\begin{center}
		\includegraphics[width=8cm]{images/lcd.jpg}
	\end{center}
	\caption{Pantalla LCD Midas 4x40}
	\label{fig:PantallaLCD}
\end{figure}

A conexión da pantalla coa placa Arduino Mega vai ser da seguinte maneira:

\begin{figure}[H]
	\begin{center}
		\includegraphics[width=15cm]{images/conexionArduinoLCD.png}
	\end{center}
	\caption{Conexión placa Arduino Mega con pantalla LCD Midas}
	\label{fig:ConexionPantalla}
\end{figure}

\begin{table}[h]
\begin{center}
\begin{tabular}{|c|c|c|}
\hline
Arduino & Pantalla Midas & Color \\
\hline
32 & DB4 & Amarelo \\
\hline
33 & DB5 & Cian\\
\hline
34 & DB6 & Azul \\
\hline
35 & DB7 & Morado \\
\hline
36 & E1 & Marrón \\
\hline
37 & E2 & Rosa \\
\hline
38 & RS & Verde \\
\hline
5V & Vdd & Vermello \\
\hline
GND & Vss & Negro \\
\hline
\end{tabular}
\caption{Conexión pines entre Arduino e Pantalla LCD Midas}
\label{TablaArduinoPantalla}
\end{center}
\end{table}

Para o control de alimentación, e, polo tanto, para o control de brillo da pantalla, vaise añadir un potenciómetro de 10k que ten o esquema reflexado na figura \ref{fig:ConexionPantalla} e os seguintes pines:

\begin{table}[htbt]
\begin{center}
\begin{tabular}{|c|c|c|}
\hline
Potenciómetro & Pantalla Midas & Arduino\\
\hline
IN & Vdd & 5V \\
\hline
OUT & V0 & -\\
\hline
\end{tabular}
\caption{Conexión do potenciómetro}
\label{TablaPotenciometro}
\end{center}
\end{table}

\subsection{Teclado Matricial}

Para poder indicarlle ó terminal que se completou unha caixa, vaise usar un teclado matricial 4x4.

\begin{figure}[H]
	\begin{center}
		\includegraphics[width=8cm]{images/teclado_storm.jpg}
	\end{center}
	\caption{STORM 720TFX SERIES *http://www.storm-interface.com/storm-720tfx-series-16-button.html}
	\label{fig:TecladoStorm}
\end{figure}

A conexión con Arduino vai ser da seguinte maneira:

\begin{figure}[H]
	\begin{center}
		\includegraphics[scale=0.75]{images/conexionArduinoKeypad.png}
	\end{center}
	\caption{Conexión placa Arduino Mega con teclado 4x4 Matrix}
	\label{fig:ConexionESP8266}
\end{figure}

\begin{table}[H]
\begin{center}
\begin{tabular}{|c|c|c|}
\hline
Arduino & Keypad Storm & Color \\
\hline
22 & 01 & Vermello \\
\hline
23 & 02 & Azul\\
\hline
24 & 03 & Blanco \\
\hline
25 & 04 & Negro \\
\hline
26 & 05 & Marrón \\
\hline
27 & 06 & Ocre\\
\hline
28 & 07 & Verde \\
\hline
29 & 08 & Morado \\
\hline
\end{tabular}
\caption{Conexión pines entre Arduino e Teclado 4x4}
\label{TablaArduinoKeypad}
\end{center}
\end{table}

\subsection{ESP8266}

É un microprocesador de baixo custo con WiFi integrado fabricado por Espressif. Supuxo unha revolución á hora de conectar o Arduino a WiFi, xa que as placas existentes eran demasiado caras, como por exemplo WiFi Shield.

Ademáis, pode comportarse como un procesador completo, con moita máis potencia que a maioría das placas Arduino.

Existen moitos modelos de placas que integran o ESP8266; o módulo ESP01 é un dos primeiros en aparecer co chip ESP8266 e un dos módulos máis sinxelos e baratos.

\begin{figure}[H]
	\begin{center}
		\includegraphics[width=12cm]{images/esp8266_conexion.jpg}
	\end{center}
	\caption{Esquema módulo ESP8266}
	\label{fig:EsquemaESP8266}
\end{figure}

En canto a comunicación WiFi, o ESP01 ten comunicación integrada 802.11  b/g/n, incluidos modos  Wi-Fi  Direct (P2P) e  soft-Ap. Inclúe unha pila de  TCP/IP completa, o que libera da maior parte do traballo de comunicación ó procesador.

\subsubsection{Esquema Eléctrico}

A conexión co módulo ESP8266 é bastante sinxela en comparación cos demáis compoñentes íntegros no proxecto. Ten os seguintes pines:

\begin{figure}[H]
	\begin{center}
		\includegraphics[width=10cm]{images/pines-esp01.png}
	\end{center}
	\caption{Pines módulo ESP8266}
	\label{fig:PinesESP8266}
\end{figure}

\begin{enumerate}
\item GND é a toma de terra.
\item GPIO2 é unha entrada/saida de propósito xeral. É o pin dixital número 2.
\item GPIO0 é unha entrada/saida de propósito xeral. É o pin dixital número 0.
\item RxD é o pin por onde se reciben os datos no porto serie. Pódese usar como pin dixital GPIO: sería o número 3.
\item TxD é o pin por onde se transmiten os datos no porto serie. Pódese usar como pin dixital GPIO: sería o número 1.
\item CH\_PD é o pin que o apaga ou encende: se está a 0V (LOW) apágase a 3,3V (HIGH), encéndese.
\item RESET é o pin que o resetea: se está a 0V (LOW) resetéase.
\item Vcc é por onde se alimenta. Funciona a 3,3 V.
\end{enumerate}

A única dificultade que imos ter vai ser á hora de alimentalo, xa que ten unha tensión de alimentación de 3,3V. En ningún caso pode alimentarse cunha tensión superior a 3,6 V, ou dañaríamolo.

A conexión coa placa vai ser a seguinte:

\begin{figure}[H]
	\begin{center}
		\includegraphics[width=10cm]{images/conexionArduinoESP8266_WiFiEsp.png}
	\end{center}
	\caption{Conexión placa Arduino Mega con módulo ESP8266}
	\label{fig:ConexionESP8266}
\end{figure}

\begin{table}[htbt]
\begin{center}
\begin{tabular}{|c|c|}
\hline
Arduino & ESP8266 \\
\hline
3,3V & Vcc \\
\hline
GND & GND \\
\hline
TX1(18) & RxD \\
\hline
Rx1(19) & TxD \\
\hline
3,3V & CH\_PD \\
\hline
\end{tabular}
\caption{Conexión pines entre Arduino e ESP8266}
\label{TablaArduinoESP8266_WiFiEsp}
\end{center}
\end{table}

\chapter{Compoñentes software}

Neste capítulo vaise dividir en dúas partes: a parte de programación do Arduino Mega 2560 e a parte da creación do servidor web con servicios API REST.

\section{Programa Arduino}

\subsection{Diagrama de fluxo}

\begin{figure}[H]
	\begin{center}
		\includegraphics[width=5cm]{images/diagrama_flujo_inicio.png}
	\end{center}
	\caption{Diagrama de fluxo do programa principal Arduino}
	\label{fig:FluxoPrincipal}
\end{figure}

\begin{figure}[H]
	\begin{center}
		\includegraphics[width=6cm]{images/diagrama_flujo_setup.png}
	\end{center}
	\caption{Diagrama de fluxo de SETUP Arduino}
	\label{fig:FluxoSETUP}
\end{figure}

\begin{figure}[H]
	\begin{center}
		\includegraphics[width=8cm]{images/diagrama_flujo_loop.png}
	\end{center}
	\caption{Diagrama de fluxo de LOOP Arduino}
	\label{fig:FluxoLOOP}
\end{figure}
 
 \subsection{Pantalla LCD}

Vaise usar a librería \textit{PantallaMidas} para manexar a parte de programación da pantalla LCD. Esta librería inicialízase pasandolle por parámetros ó constructor os pines que está conectado. Faise da seguinte maneira:

\begin{minted}{cpp}
PantallaMidas pantalla(DB4, DB5, DB6, DB7, E1, E2, RS);
\end{minted}

Para inicializar a pantalla LCD, débese configurala. Neste caso vaise usar a configuración de bus de 4 bits e ten os seguintes pasos reflexados na Figura.

\begin{figure}[H]
	\begin{center}
		\includegraphics[scale=0.6]{images/inicializar_lcd.png}
	\end{center}
	\caption{Diagrama de Flujo iniciación pantalla LCD Midas}
	\label{fig:DiagramaFlujoPantalla}
\end{figure}

Para elo, usarase o metodo \textit{configura} como se mostra a continuación:
\begin{minted}{cpp}
pantalla.configura();
\end{minted}

\subsection{Teclado Matricial}

Para usar este compoñente, elaborouse unha clase en C++ que permite a identificación da tecla pulsada.

Para poder usar esta clase, hai que declarar un constructor pasandolle por referencia os pines e un String cos caracteres asociados. Un exemplo é o seguinte:

\begin{minted}{cpp}
Teclado4x4 teclado(F1, F2, F3, F4, C1, C2, C3, C4, caracteres);
\end{minted}

Unha vez feito o constructor, a función \textit{configura} vai configurar o teclado para poder manexalo.

\begin{minted}{cpp}
teclado.configura();
\end{minted}

Esta clase ten a función chamada \textit{comprueba()} que comproba se se pulsou unha determinada tecla devolvendo o caracter pulsado. Un exemplo:

\begin{minted}{cpp}
char caracter = teclado.comprueba();
\end{minted}


\subsection{ESP6288 ESP-01}

\subsubsection{Comandos AT}

Serven para establecer comunicación entre o módulo e a placa Arduino. Os comandos básicos refléxanse na seguinte táboa:

\begin{table}[!h]
\centering
\resizebox{16cm}{!} {
\begin{tabular}{|c|c|c|c|c|}
\hline
Comando & Descripción & Resposta & Configuración & Parámetros \\
\hline
AT & Test de inicio & OK & - & - \\
\hline
AT+RST & Restea o módulo & Module info & - & -\\
\hline
AT+GMR & Devolve a versión \newline do módulo & Fw Version & - & - \\
\hline
AT+CWMODE & Configura o  \newline modo WiFi & Mode set & AT+CWMODE=(Modo) & Modo 1=Sta, 2=AP, 3=both\\
\hline
AT+CWLAP & Lista todos os Puntos de Acceso (AP) & list of AP & - & - \\
\hline
AT+CWJAP & Conéctase a AP & OK & AT+CWJAP=(ssid),(pwd) & ssid= Nombre red de AP, pwd= contraseña de AP\\
\hline
AT+CWQAP & Finaliza a conexión con AP &  & AT+CWQAP & \\
\hline
AT+CWSAP & Configura os \newline parametros da AP & & AT+CWSAP=(ssid),(pwd), \newline (chl),(ecn) & ssid, contraseña, \newline canal, encriptacion \\
\hline
AT+CWLIF & Comproba a \newline conexión IP do módulo &  & AT+CWLIF &  \\
\hline
AT+CIPSTATUS & Comproba a conexión & (id),(tipo),(direccion),(porto) & AT+CIPSTATUS & -   \\
\hline
\end{tabular}
}
\caption{Comandos básicos AT}
\label{comandosAT}
\end{table}

Para o manexo dos comandos, vaise usar unha librería Open Source \textit{WiFiEsp}\cite{esp}.

\subsubsection{Conexión e consultas contra o servidor}

Implementouse a clase \textit{ConnectEsp}, que se usará para a conexión a un punto de acceso (AP) como para facer consultas a base de datos creada por Django API REST. 

Para manexar os datos recibidos das consultas, vaise usar outra librería Open Source chamada \textit{ArduinoJSON}, que vai permitir tanto xerar como dividir datos tipo Json \cite{Json}.

\section{Django RESTful Web Framework}

É unha librería que nos vai permitir construir un API REST sobre Django. Ofrece unha diversidade de métodos e funcións para a xestión dos recursos.

Esta librería é totalmente portable; podese usar tanto con Windows, Linux ou Mac OSX. Neste proxecto vaise usar o sistema operativo MacOs Hight Sierra.

A continuación móstrase como creala:

\subsection{Crear directorio base}

Vaise crear un directorio onde vai estar contido o código. Ábrese unha ventana de terminal:

\begin{minted}{bash}
  $ mkdir tfg_django
  $ cd tfg_django
\end{minted}

\subsection{Configurar entorno virtual}

Esto vai permitir aislar as dependencias do proxecto, eliminando os conflictos de librerías locais do sistema.
\begin{minted}{bash}
  $ virtualenv tutorial
  $ source tfg_django/bin/activate #esto activará o noso entorno virtual.
\end{minted}

A continuación, instalaranse os paquetes necesarios.

\begin{minted}{bash}
  $ pip install django #instalar o paquete de django
  $ pip install djangorestframework #instalar o paquete de Django Rest Framework
\end{minted}

\subsection{Crear proxecto e configuración inicial}

Vaise crear o proxecto dentro do directorio base e inicializar a configuración. A API vaise chamar \textit{almacen} e vaise crear unha aplicación que se chamará \textit{encargos}.

\begin{minted}{bash}
  $ django-admin.py startproject almacen
  $ cd encargos
  $ django-admin.py startapp encargos
\end{minted}

Vaise modificar o arquivo \textit{almacen/settings.py} para añadir a aplicación creada mais a librería.

\begin{minted}{python}
INSTALLED_APPS = (
    ...
    'rest_framework',
    'encargos',
)
\end{minted}

\subsection{Crear Models}

Vaise modificar o arquivo \textit{encargos/models.py} para crear o noso modelo.

\begin{minted}{python}
from django.db import models

class Encargo(models.Model):

    operario = models.ForeignKey(
        'auth.User',
        related_name='encargo',
        on_delete=models.CASCADE)
    created = models.DateTimeField(auto_now_add=True)
    completado = models.BooleanField(default=False)

    class Meta:
        ordering = ('created',)


class Producto(models.Model):
    AREA1 = '1'
    AREA2 = '2'
    AREAS = (
        (AREA1, 'Area 1'),
        (AREA2, 'Area 2'),
    )
    name = models.CharField(max_length=50)
    localizacion = models.CharField(
        max_length=2,
        choices=AREAS,
        default=AREA1, 
    )

    class Meta:
        ordering = ('name',)

    def __str__(self):
        return self.name


class Detalle_Encargo(models.Model):
    producto = models.ForeignKey(
        Producto, 
        related_name='producto', 
        on_delete=models.CASCADE)
    encargo = models.ForeignKey(
        Encargo, 
        on_delete=models.CASCADE)
    cantidade = models.IntegerField()
)
\end{minted}

Para crear e xerar migracións usaremos o comando:

\begin{minted}{bash}
  $ python manage.py makemigrations 
  $ python manage.py migrate
\end{minted}

\subsection{Crear ModelSerializers}

Créase un serializer en \textit{encargos/serializers.py}

\begin{minted}{python}
from rest_framework import serializers
from encargos.models import Encargo
from encargos.models import Producto
from encargos.models import Detalle_Encargo
from django.contrib.auth.models import User


class UserEncargoSerializer(serializers.HyperlinkedModelSerializer):
    class Meta:
        model = Encargo
        fields = (
            'url',
            'name')


class UserSerializer(serializers.HyperlinkedModelSerializer):
    encargos = UserEncargoSerializer(many=True, read_only=True)

    class Meta:
        model = User
        fields = (
            'url', 
            'pk',
            'username',
            'encargos',
            'password'
        )

class EncargoSerializer(serializers.HyperlinkedModelSerializer):

    class Meta:
        model = Encargo
        fields = (
            'url',
            'pk',
            'created',
            'completado'
    )


class ProductoSerializer(serializers.HyperlinkedModelSerializer):
    localizacion = serializers.ChoiceField(choices=Producto.AREAS)
    area_description = serializers.CharField(source='get_area_display', read_only=True)
    class Meta:
        model = Producto
        fields = (
            'url',
            'pk',
            'name',
            'localizacion',
            'area_description',
        )

class Detalle_EncargoSerializer(serializers.ModelSerializer):
    localizacion_producto = ProductoSerializer(many=True, read_only=True)
    encargo_completado = EncargoSerializer(many=True, read_only=True)
    class Meta:
        model = Detalle_Encargo
        fields = (
            'url',
            'pk',
            'encargo',
            'producto',
            'cantidade',
            'localizacion_producto',
            'encargo_completado'
        )
\end{minted}

\subsection{Vistas normales en Django}

Unha vez creado o serializador, vaise crear a API usando vistas funcionais ou vistas simples en Django. Para eso, vaise modificar o arquivo \textit{encargos/views.py}.

\begin{minted}{python}
from encargos.models import Encargo
from encargos.models import Producto
from encargos.models import Detalle_Encargo
from encargos.serializers import EncargoSerializer
from encargos.serializers import ProductoSerializer
from encargos.serializers import Detalle_EncargoSerializer
from rest_framework import generics
from rest_framework.response import Response
from rest_framework.reverse import reverse
from django.contrib.auth.models import User
from encargos.serializers import UserSerializer
from rest_framework import permissions
from encargos.permissions import IsOperarioOrReadOnly


class UserList(generics.ListAPIView):
    queryset = User.objects.all()
    serializer_class = UserSerializer
    name = 'user-list'


class UserDetail(generics.RetrieveAPIView):
    queryset = User.objects.all()
    serializer_class = UserSerializer
    name = 'user-detail'

class EncargoList(generics.ListCreateAPIView):
    queryset = Encargo.objects.all()
    serializer_class = EncargoSerializer
    name = 'encargo-list'
    permission_classes = (
        permissions.IsAuthenticatedOrReadOnly,
        IsOperarioOrReadOnly,
        )
    def perform_create(self, serializer):
        # Pass an additional owner field to the create method
        # To Set the owner to the user received in the request
        serializer.save(operario=self.request.user)


class EncargoDetail(generics.RetrieveUpdateDestroyAPIView):
    queryset = Encargo.objects.all()
    serializer_class = EncargoSerializer
    name = 'encargo-detail'
    permission_classes = (
        permissions.IsAuthenticatedOrReadOnly,
        IsOperarioOrReadOnly)


class ProductoList(generics.ListCreateAPIView):
    queryset = Producto.objects.all()
    serializer_class = ProductoSerializer
    name = 'producto-list'


class ProductoDetail(generics.RetrieveUpdateDestroyAPIView):
    queryset = Producto.objects.all()
    serializer_class = ProductoSerializer
    name = 'producto-detail'
    

class Detalle_EncargoList(generics.ListCreateAPIView):
    queryset = Detalle_Encargo.objects.all()
    serializer_class = Detalle_EncargoSerializer
    name = 'detalle_encargo-list'


class Detalle_EncargoDetail(generics.RetrieveUpdateDestroyAPIView):
    queryset = Detalle_Encargo.objects.all()
    serializer_class = Detalle_EncargoSerializer
    name = 'detalle_encargo-detail'
    

class ApiRoot(generics.GenericAPIView):
    name = 'api-root'
    def get(self, request, *args, **kwargs):
        return Response({
            'productos': reverse(ProductoList.name, request=request),
            'detalle_encargo': reverse(Detalle_EncargoList.name, request=request),
            'encargos': reverse(EncargoList.name, request=request),
            'users': reverse(UserList.name, request=request),
            })
\end{minted}

Finalmente, agrégase as url para poder mostrar os servicios. Modificase o arquivo \textit{encargos/urls.py}.
\begin{minted}{python}
from django.conf.urls import url
from encargos import views


urlpatterns = [
    url(r'^encargos/$', 
        views.EncargoList.as_view(),
        name=views.EncargoList.name),
    url(r'^encargos/(?P<pk>[0-9]+)/$', 
        views.EncargoDetail.as_view(),
        name=views.EncargoDetail.name),
    url(r'^productos/$', 
        views.ProductoList.as_view(),
        name=views.ProductoList.name),
    url(r'^productos/(?P<pk>[0-9]+)/$', 
        views.ProductoDetail.as_view(),
        name=views.ProductoDetail.name),
    url(r'^detalle_encargo/$', 
        views.Detalle_EncargoList.as_view(),
        name=views.Detalle_EncargoList.name),
    url(r'^detalle_encargo/(?P<pk>[0-9]+)/$', 
        views.Detalle_EncargoDetail.as_view(),
        name=views.Detalle_EncargoDetail.name),
    url(r'^users/$',
        views.UserList.as_view(),
        name=views.UserList.name),
    url(r'^users/(?P<pk>[0-9]+)/$',
        views.UserDetail.as_view(),
        name=views.UserDetail.name),
    url(r'^$',
        views.ApiRoot.as_view(),
        name=views.ApiRoot.name),
]
\end{minted}

\subsection{Visualización API}

Para mostrar a API, levantarase o servidor con:

\begin{minted}{bash}
  $ python manage.py runserver 
\end{minted}

Por defecto, o servidor lanzarase en \textit{http://127.0.0.0.1:8000}. Esto pódese modificar pasándolle como argumento o host e porto desexado. Por exemplo:

\begin{minted}{bash}
  $ python manage.py runserver 192.168.1.116:8000
\end{minted}

\chapter{Resultados}

Para lanzar o servidor, mediante unha ventana de terminal escribiremos:

\begin{minted}{bash}
  $ cd tfg_django                                 #accédese a carpeta do proxecto
  $ source env/bin/activate                       #actívase o entorno virtual
  $ cd almacen                                    #accédese a app do servicio API
  $ python manage.py runserver 192.168.100.3:8000 #lánzase o servidor no host do ordenador
\end{minted}

Conectarase a placa Arduino Mega 2560 á unha fonte de alimentación. Neste caso, conectarase vía USB. Unha vez conectado, terminal de operador inalámbrico vai mostrar que se conectou á rede inalámbrica asignada. Neste caso conectarase a \textit{HUAWEI-CW4a}.

\begin{figure}[H]
	\begin{center}
		\includegraphics[width=6cm]{images/conectar_rede.JPG}
	\end{center}
	\caption{Conexión á rede HUAWEI-Cw4a}
	\label{fig:Encargo1}
\end{figure}

O terminal de operador mostra se hai encargos para facer. Neste caso, vai reflexar o encargo con id 2, xa que como podemos ver, o encargo con id 1 está completado.

\begin{figure}[H]
	\begin{center}
		\includegraphics[width=5cm]{images/encargo_1.png}
	\end{center}
	\caption{Encargo con pk = 1}
	\label{fig:Encargo1}
\end{figure}

\begin{figure}[H]
	\begin{center}
		\includegraphics[width=5cm]{images/id_encargo.JPG}
	\end{center}
	\caption{Encargo pk = 2 no terminal}
	\label{fig:Encargo1}
\end{figure}



Ó darlle o operador a tecla ENT, o terminal mostrará o primeiro producto co seu nome, localización e cantidade. 

\begin{figure}[H]
	\begin{center}
		\includegraphics[width=6cm]{images/detalles_producto.JPG}
	\end{center}
	\caption{Detalles Producto}
	\label{fig:Encargo1}
\end{figure}

Cando o operador meta ese producto ou productos na caixa e lle de a tecla ENT, o terminal mostrará o seguinte producto e así sucesivamente.

Cando se complete o pedido, o terminal mostrará unha mensaxe na pantalla de que se rematou, quedando rexistrado na API.
\begin{figure}[H]
	\begin{center}
		\includegraphics[width=8cm]{images/API_modificada.png}
	\end{center}
	\caption{Producto completado na aplicación}
	\label{fig:Encargo1}
\end{figure}
\begin{figure}[H]
	\begin{center}
		\includegraphics[width=8cm]{images/encargo_completado.JPG}
	\end{center}
	\caption{Producto completado no terminal}
	\label{fig:Encargo1}
\end{figure}

En todo momento se poderá visualizar a parte do servidor cunha interfaz sencilla pero efectiva que será a seguinte:

\begin{figure}[H]
	\begin{center}
		\includegraphics[width=12cm]{images/API_Root_Django.png}
	\end{center}
	\caption{Interfaz Django API RESTful}
	\label{fig:Encargo1}
\end{figure}

\chapter{Líneas futuras}

Para futuras versións deste proxecto, contémplanse as seguintes melloras na parte de sofware:

\begin{itemize}
    \item Mellora da interfaz do servicio web, podendo subir fotos dos productos.
    \item Autentificación do operador mediante uso dun módulo que reconoza as huellas dactilares ou por código de operador.
\end{itemize}


\begin{thebibliography}{99}

\bibitem{IoT} \textsc{Brian Russell, Drew Van Duren}; \textit{Practical Internet of Things Security}, PACKT Publishing (Xuño 2016).
\bibitem{Django} \textsc{Gastón C. Hillar}, \emph{Building RESTful Python Web}, PACKT Publishing (Outubro 2016).
\bibitem{esp} \textit{WifiEsp Library}, A ESP8266 Library for Arduino. Dispoñible en: \url{https://github.com/bportaluri/WiFiEsp}.
\bibitem{Json} \textit{Arduino JSON}, A JSON Library for Arduino. Dispoñible en: \url{https://arduinojson.org}.

					
\end{thebibliography}

\stopcontents[parts]

\cleardoublepage
%%%%%%%%%%%%%%%%%%%%%%%%%%%%%%%%%%%%%%%%%%%%%%%%%%%%%%%%%%%%%%%%%%%%%%%%%%%%%%%%%%%%%%%%%%%%%%%%%%%%%%%%%%%%%%%%%%%%%%%%%%%%%%%%%%%%%%%%%%%%%%%%%%%%%%%%%%%%%%%%%%%%%%%%%%%%%
%%%%%%%%%%%%%%%%%%%%%%%%%%%%%%%%%%%%%%%%%%%%%%%%%%%%%%%%%%%%%%%%%%%%%%%%%%%%%%%%%%%%%%%%%%%%%%%%%%%%%%%%%%%%%%%%%%%%%%%%%%%%%%%%%%%%%%%%%%%%%%%%%%%%%%%
% DOCUMENTO ANEXOS
\renewcommand{\documento}{ANEXOS}
\begin{tikzpicture}[remember picture, overlay]
  \draw[line width = 0.5pt] ($(current page.north west) + (-20pt,-800.5pt)$) rectangle ($(current page.south east) + (-133.5pt,-722.5pt)$);
\end{tikzpicture}

\begin{center}
\begin{figure}[htbp]
\begin{center}
\includegraphics[angle=0, height=3.8cm]{images/EEILogo.png}
\end{center}
\end{figure}
\ \\
\begin{large}
\begin{center}
\color{blue}\textbf{Escola de Enxeñería Industrial}
\end{center}
\end{large}
\ \\
\ \\
\begin{large}
\begin{center}
\textbf{TRABALLO FIN DE GRAO}
\end{center}
\end{large}
\ \\
\ \\
\begin{large}
\begin{center}
{\titulouno}
\end{center}
\end{large}
\ \\
\ \\
\begin{normalsize}
\begin{center}
\textbf{\grado}
\end{center}
\end{normalsize}
\ \\
\ \\
\ \\
\ \\
\begin{normalsize}
\begin{center}
\textbf{Documento}
\end{center}
\end{normalsize}
\ \\
\begin{normalsize}
\begin{center}
\part{\bf{ANEXOS}}\thispagestyle{empty}
\end{center}
\end{normalsize}
\ \\
\ \\
\ \\
\ \\

\begin{center}
\begin{figure}[htbp]
\begin{center}
\includegraphics[angle=0, height=0.8cm]{images/UVIGOLogo.png}
\end{center}
\end{figure}
\end{center}

\end{center}

\cleardoublepage


\pagestyle{fancy}
%%%%%%%%%%%%%%%%%%%%%%%%%%%%%%%%%%%%%%%%%%%%%%%%%%%%%%%%%%%%%%%%%%%%%%%%%%%%%%%%%%%%%%%%%%%%
%%%%%%%%%%%%%%%%%%%%%%%%%%%%%%%%%%%%%%%%%%%%%%%%%%%%%%%%%%%%%%%%%%%%%%%%%%%%%%%%%%%%%%%%%%%%%%%%%%%%%%%%%%%%%%%%%%%%%%%%%%%%%%%%%%%%%%%%%%%%%%%%%%%%%%%%%%%%%%%%%%%%%%%%%%%%%
\addcontentsline{toc}{section}{Índice del documento Anexos}
\startcontents[parts]
\cleardoublepage

\begin{center}{\large \bf Índice de ANEXOS}\end{center}

{\hypersetup{hidelinks}\printcontents[parts]{}{-1}{\setcounter{tocdepth}{5}}}

\cleardoublepage
\chapter{Códigos de programación do Arduino}

\section{Librerías}

\subsection{PantallaMidas}

\subsubsection{Cabeceira}
\begin{lstlisting}
#ifndef PantallaMidas_h
#define PantallaMidas_h

#include <Arduino.h>

class PantallaMidas {
// Clase para el manejo de una pantalla LCD alfanumérica de 4 filas y 40 columnas modelo
// Midas MC44005A6W-FPTLW desde una placa Arduino. Se configura un bus de datos de 4 bits. 

  private:  // Recursos privados a esta clase
  
    int _DB4, _DB5, _DB6, _DB7, _E1, _E2, _RS;
    // Número de señales Arduino utilizadas para la conexión a la pantalla
    
    int _fila;  
    // Última fila posicionada 1...4
    
    void enviaOrden1(unsigned char orden);
    // Envía una orden al controlador 1 de la pantalla
    
    void enviaOrden2(unsigned char orden);
    // Envía una orden al controlador 2 de la pantalla
    
    void enviaOrden12(unsigned char orden);
    // Envía una orden a los controladores 1 y 2 de la pantalla
    
    void busDatos4Bits(unsigned char dato);
    // Establece un valor para el bus de datos, utilizando los 4 bits
    // menos significativos del parámetro
    
    void pulsoE1();
    // Pulso a nivel alto en señal E1 para transferencia con el controlador 1
    
    void pulsoE2();
    // Pulso a nivel alto en señal E1 para transferencia con el controlador 1

  public:  // Miembros públicos de esta clase
  
    PantallaMidas(int DB4, int DB5, int DB6, int DB7, int E1, int E2, int RS);
    // Constructor al que se le indica qué señales Arduino están conectadas a las 
    // señales de la pantalla
    
    void configura();
    // Método que hay que utilizar en setup() para configurar e inicializar los 
    // controladores de la pantalla
    
    void borra();
    // Borra toda la información mostrada en la pantalla
    
    void posiciona(int fila, int columna);
    // Posiciona el cursor en una fila (1...4) y columna (1...40)
    
    void escribeCadena(const char * cadena);
    // Escribe una cadena de caracteres en la posición actual del cursor

    void escribeCadena(const String cadena);
    
    void escribeCaracter(char caracter);
    // Escribe un carácter en la posición actual del cursor

    void muestraCursor(int muestralo);
    // Hace que el cursor sea visible en función del buleano que se pasa por parámetro

};

#endif
\end{lstlisting}

\subsubsection{Código}

\begin{lstlisting}
#include <PantallaMidas.h>

PantallaMidas::PantallaMidas(int DB4, int DB5, int DB6, int DB7, int E1, int E2, int RS) {  
// Constructor al que se le indica qué señales Arduino están conectadas a las 
// señales de la pantalla

  _DB4 = DB4;
  _DB5 = DB5;
  _DB6 = DB6;
  _DB7 = DB7;
  _E1 = E1;
  _E2 = E2;
  _RS = RS;
  // Guarda los parámetros en datos privados
  
  _fila = 1;  // La fila inicial es la primera
}

void PantallaMidas::configura() {
// Método que hay que utilizar en setup() para configurar e inicializar los 
// controladores de la pantalla

  pinMode(_DB4, OUTPUT);
  pinMode(_DB5, OUTPUT);
  pinMode(_DB6, OUTPUT);
  pinMode(_DB7, OUTPUT);
  pinMode(_E1, OUTPUT);
  pinMode(_E2, OUTPUT);
  pinMode(_RS, OUTPUT);
  // Configura todas las señales como salidas digitales

  digitalWrite(_E1, LOW);
  digitalWrite(_E2, LOW);
  digitalWrite(_RS, LOW);
  // Pone a nivel bajo todas estas señales

  // A continuación se ejecuta la secuencia de inicialización de ambos controladores
  
  delay(20);  // Espera 20 ms

  busDatos4Bits(3);
  pulsoE1();
  pulsoE2();
  // Transfiere a ambos controladores el valor 3 

  delay(5);  // Espera 5 ms

  busDatos4Bits(3);
  pulsoE1();
  pulsoE2();
  // Transfiere a ambos controladores el valor 3 
  
  delay(1);  // Espera 1 ms

  busDatos4Bits(3);
  pulsoE1();
  pulsoE2();
  // Transfiere a ambos controladores el valor 3 
  
  delay(1);  // Espera 1 ms
  
  busDatos4Bits(2);
  pulsoE1();
  pulsoE2();
  // Transfiere a ambos controladores el valor 2 

  delay(1);  // Espera 1 ms

  enviaOrden12(0x2D);
  delay(1);
  //0000101101
  //0 0 0 0 1 DL N F - -
  // Function set: 
  // DL=0 establece bus de 4bits 
  // N=1 establece manejo de 2 filas
  // F=1 establece juego de caracteres de 5x11 puntos
  
  enviaOrden12(0x08);
  delay(1);
  // Display off

  borra();
  delay(2);
  // Clear display
  
  enviaOrden12(0x06);
  delay(1);
  // 0 0 0 0 0 0 0 0 0 1 ID SH
  // Entry mode set: 
  // ID=1 para que el cursor se mueva hacia derecha
  // SH=0 para que no haya desplazamiento de la pantalla
  
  enviaOrden12(0x0C);  // Display on
  delay(1);
}


void PantallaMidas::muestraCursor(int muestralo) {
// Hace que el cursor sea visible en función del buleano que se pasa por parámetro

  unsigned char orden1, orden2;
  
  if (muestralo) {  // Si hay que mostrarlo ...
    if (_fila < 3) {  // Si es para el controlador 1 ...
      orden1 = 0x0F;  // Muestra cursor en 1
      orden2 = 0x0C;  // Oculta cursor en 2
    } else {
      orden1 = 0x0C;  // Oculta cursor en 1
      orden2 = 0x0F;  // Muestra cursor en 2
    }
  } else {  // Si hay que ocultarlo ...
    orden1 = 0x0C;  // Oculta cursor en 1
    orden2 = 0x0C;  // Oculta cursor en 2
  }
  enviaOrden1(orden1);  // Envía la orden al controlador 1
  enviaOrden2(orden2);  // Envía la orden al controlador 2
}


void PantallaMidas::pulsoE1() {
// Pulso a nivel alto en señal E1 para transferencia con el controlador 1

  digitalWrite(_E1, HIGH);
  digitalWrite(_E1, LOW);
}

void PantallaMidas::pulsoE2() {
// Pulso a nivel alto en señal E2 para transferencia con el controlador 2

  digitalWrite(_E2, HIGH);
  digitalWrite(_E2, LOW);
}


void PantallaMidas::busDatos4Bits(unsigned char dato) {
// Establece un valor para el bus de datos, utilizando los 4 bits
// menos significativos del parámetro

  if (dato & 0x01)  // Si el bit 0 es un 1, entonces ...
      digitalWrite(_DB4, HIGH);  // pon la señal _DB4 a nivel alto
      else digitalWrite(_DB4, LOW);  // si no, pon la señal _DB4 a nivel bajo
      
  if (dato & 0x02)  // Si el bit 1 es un 1, entonces ...
      digitalWrite(_DB5, HIGH);  // pon la señal _DB5 a nivel alto
      else digitalWrite(_DB5, LOW);  // si no, pon la señal _DB5 a nivel bajo
      
  if (dato & 0x04)  // Si el bit 2 es un 1, entonces ...
      digitalWrite(_DB6, HIGH);  // pon la señal _DB6 a nivel alto
      else digitalWrite(_DB6, LOW);  // si no, pon la señal _DB6 a nivel bajo
      
  if (dato & 0x08)  // Si el bit 3 es un 1, entonces ...
      digitalWrite(_DB7, HIGH);  // pon la señal _DB7 a nivel alto
      else digitalWrite(_DB7, LOW);  // si no, pon la señal _DB7 a nivel bajo
}


void PantallaMidas::borra() {
// Borra toda la información mostrada en la pantalla

  enviaOrden12(0x01); 
  // Envía la orden de código 0x01 a los dos controladores
  
  delay(2);  // Espera 2 ms a que se termine de ejecutar
}


void PantallaMidas::escribeCadena(const char * cadena) {
// Escribe una cadena de caracteres en la posición actual del cursor

  int i;  // Contador para el bucle
  
  i = 0;  // Desde el comienzo de la cadena ...
  while(cadena[i])  // Mientras no se llegue a un byte con valor 0 ...
    escribeCaracter(cadena[i++]);  // Escribe el carácter y pasa al siguiente
}

void PantallaMidas::escribeCadena(const String cadena) {
  //Escribe una cadena de string en la poisicion actual del cursor

  int i;

  i = 0;
  while(cadena[i])
    escribeCaracter(cadena[i++]);
}


void PantallaMidas::posiciona(int fila, int columna) {
// Posiciona el cursor en una fila (1...4) y columna (1...40)

  int posicion;  // Posición en la memoria interna del controlador

  if (fila < 3) {   // Si es para el primer controlador ...
    
    posicion = (fila - 1) * 0x40 + columna - 1;
    // La primera posición comienza en 0 para la fila 1 y en 0x40 para la fila 2
    // Al avanzar en las columnas, se avanza en memoria a posiciones consecutivas
    
    enviaOrden1(0x80 | posicion);
    // Envía una orden al controlador 1 con la nueva posición del cursor
    
  } else {  // Lo mismo para el controlador 2
    posicion = (fila - 3) * 0x40 + columna - 1;
    enviaOrden2(0x80 | posicion);
  }
  
  delay(1);  // Espera 1 ms a que el cursor se posicione
  _fila = fila;  // Recuerda la fila donde está situado el cursor
}


void PantallaMidas::enviaOrden1(unsigned char orden) {
// Envía una orden al controlador 1 de la pantalla

  digitalWrite(_RS, LOW);  // Pone la señal RS a nivel bajo
  
  busDatos4Bits(orden >> 4);  // Pone los 4 bits más significativos en el bus de datos
  pulsoE1();  // Pulso en E1 para transferir esos bits
  busDatos4Bits(orden & 0x0F);  // Pone los 4 bits menos significativos en el bus de datos
  pulsoE1();  // Pulso en E1 para transferir esos bits
}


void PantallaMidas::enviaOrden2(unsigned char orden) {
// Envía una orden al controlador 2 de la pantalla

  digitalWrite(_RS, LOW);  // Pone la señal RS a nivel bajo
  
  busDatos4Bits(orden >> 4);  // Pone los 4 bits más significativos en el bus de datos
  pulsoE2();  // Pulso en E2 para transferir esos bits
  busDatos4Bits(orden & 0x0F);  // Pone los 4 bits menos significativos en el bus de datos
  pulsoE2();  // Pulso en E2 para transferir esos bits
}


void PantallaMidas::enviaOrden12(unsigned char orden) {
// Envía una orden a ambos controladores

  enviaOrden1(orden);   // Envía la orden al controlador 1
  enviaOrden2(orden);   // Envía la orden al controlador 2
}


void PantallaMidas::escribeCaracter(char caracter) {
// Envía un carácter a la pantalla, que se mostrará en la posición actual del cursor. 
// La posición del cursor se incrementa

  digitalWrite(_RS, HIGH);   // Pone la señal RS a nivel alto
  
  busDatos4Bits(caracter >> 4);  // Pone los 4 bits más significativos en el bus de datos
  if (_fila < 3)  // Si es para el controlador 1 ...
    pulsoE1();   // pulso en E1 para transferir esos bits al controlador 1
    else pulsoE2();  // si no, pulso en E2 para transferir esos bits al controlador 2
    
  busDatos4Bits(caracter & 0x0F);    // Pone los 4 bits menos significativos en el bus de datos
  if (_fila < 3)  // Si es para el controlador 1 ...
    pulsoE1();   // pulso en E1 para transferir esos bits al controlador 1
    else pulsoE2();  // si no, pulso en E2 para transferir esos bits al controlador 2
    
  delay(1);  // Retardo de 1 ms para esperar a que se escriba el carácter
}
\end{lstlisting}

\subsection{Teclado 4x4}

\subsubsection{Cabeceira}

\begin{lstlisting}
#ifndef Teclado4x4_h
#define Teclado4x4_h

class Teclado4x4 {
  private:
    int _salidasFilas[4], * _entradasColumnas[4];
    char _caracteres[17];
    int _pulsado, _fila, _columna;
    unsigned long _t;
  public:
    Teclado4x4(int F1, int F2, int F3, int F4, int C1, int C2, int C3, int C4, char *caracteres);
    void configura();
    int comprueba();
};

#endif
\end{lstlisting}

\subsubsection{Codigo}
\begin{lstlisting}
#include <Teclado4x4.h>
#include <Arduino.h>

Teclado4x4::Teclado4x4(int F1, int F2, int F3, int F4, int C1, int C2, int C3, int C4, char *caracteres) {
  _salidasFilas[0] = F1;
  _salidasFilas[1] = F2;
  _salidasFilas[2] = F3;
  _salidasFilas[3] = F4;
  _entradasColumnas[0] = C1;
  _entradasColumnas[1] = C2;
  _entradasColumnas[2] = C3;
  _entradasColumnas[3] = C4;
  strcpy(_caracteres,caracteres);
  // Guarda los parámetros en datos privados
}


void Teclado4x4::configura() {
  int i;
  //memcpy (_salidasFilas, salidasFilas, 4 * sizeof(int));
  //memcpy (_entradasColumnas, entradasColumnas, 4 * sizeof(int));
  //strcpy (_caracteres, caracteres);
  for (i = 0; i < 4; i++)
    pinMode(_entradasColumnas[i], INPUT_PULLUP);
  _pulsado = 0;
  _t = millis();
}


int Teclado4x4::comprueba () {
  int iSalida, iEntrada;

  if (millis() - _t < 10) 
    return 0;
    else _t = millis();

  if (_pulsado) {
    pinMode(_salidasFilas[_fila], OUTPUT);
    digitalWrite(_salidasFilas[_fila], LOW);
    _pulsado = digitalRead(_entradasColumnas[_columna]) == LOW;
    pinMode(_salidasFilas[_fila], INPUT);    
    if (_pulsado)
      return 0;
  }
  
  if (! _pulsado) {
    for (iSalida = 0; iSalida < 4 && ! _pulsado; iSalida ++) {
      pinMode(_salidasFilas[iSalida], OUTPUT);
      digitalWrite(_salidasFilas[iSalida], LOW);
      for (iEntrada = 0; iEntrada < 4 && ! _pulsado; iEntrada ++) {
        if (digitalRead(_entradasColumnas[iEntrada]) == LOW) {
          _pulsado = 1;
          _fila = iSalida;
          _columna = iEntrada;
        }
      }
      pinMode(_salidasFilas[iSalida], INPUT);
    }
  }
  
  if (_pulsado)
    return _caracteres[_fila * 4 + _columna];
  else return 0;
}
\end{lstlisting}

\subsection{Consola}
\subsubsection{Cabeceira}
\begin{lstlisting}
#ifndef CONSOLA_H_
#define CONSOLA_H_

#include "PantallaMidas.h"
#include "Teclado4x4.h"

class Consola {
  // Un objeto de esta clase representa a una consola compuesta por un teclado y una pantalla
  
  private:
      PantallaMidas pantalla;
      Teclado4x4 teclado;
      // Cada objeto Consola contiene objetos para manejar el teclado y la pantalla
      
      char _cadena[41];
      
  public: 
  
      Consola(int DB4, int DB5, int DB6, int DB7, int E1, int E2, int RS, int F1, int F2, int F3, int F4, int C1, int C2, int C3, int C4, char * caracteres);
      // Constructor al que le proporcionamos los datos necesarios para inicializar la pantalla y el teclado
      
      void configura();  // Utilizado en el setup() para configurar estos dispositivos
      
      void visualizaEntero(int fila, int columna, unsigned int valor);
      void visualizaReal(int fila, int columna, float valor, int nCaracteres, int nDecimales);
      void visualizaCadena(int fila, int columna, char * cadena);  
      void visualizaCadena(int fila, int columna, String cadena);
      // Para visualizar información en pantalla
      
      char introduceCaracter();
      int introduceEntero(int fila, int columna, int nCaracteres);
      float introduceReal(int fila, int columna, int nCaracteres);
      void introduceCadena(int fila, int columna, char * validos, int nCaracteres, char * resultado);
      // Para introducir información por teclado
      
      void borraPantalla();
      // Para borrar la pantalla
};

#endif
\end{lstlisting}

\subsubsection{Código}
\begin{lstlisting}
#include "Consola.h"
#include <string.h>


Consola::Consola(int DB4, int DB5, int DB6, int DB7, int E1, int E2, int RS, int F1, int F2, int F3, int F4, int C1, int C2, int C3, int C4, char * caracteres):
    pantalla(DB4, DB5, DB6, DB7, E1, E2, RS),
    teclado(F1, F2, F3, F4, C1, C2, C3, C4, caracteres) {
}

void Consola::configura() {
  pantalla.configura();
  teclado.configura();
}

void Consola::visualizaEntero(int fila, int columna, unsigned int valor) {
  pantalla.posiciona(fila, columna);
  unsigned long aux = valor;
  //itoa(valor, _cadena, 10);
  sprintf(_cadena,"%ld",aux);
  pantalla.escribeCadena(_cadena);
}

void Consola::visualizaReal(int fila, int columna, float valor, int nCaracteres, int nDecimales) {
  pantalla.posiciona(fila, columna);
  dtostrf(valor, nCaracteres, nDecimales, _cadena);
  pantalla.escribeCadena(_cadena);
}

void Consola::visualizaCadena(int fila, int columna, char * cadena) {
  pantalla.posiciona(fila, columna);
  pantalla.escribeCadena(cadena);
}

void Consola::visualizaCadena(int fila, int columna, String cadena) {
  pantalla.posiciona(fila, columna);
  pantalla.escribeCadena(cadena);
}

char Consola::introduceCaracter() {
  char caracter;
  do {
    caracter = teclado.comprueba();
  } while (caracter == 0);
  return caracter;
}

void Consola::introduceCadena(int fila, int columna, char * validos, int nCaracteres, char * resultado) {
  int i;
  char caracter;
  pantalla.posiciona(fila, columna);
  for (i = 0; i < nCaracteres; i++)
      pantalla.escribeCaracter(' ');
  pantalla.posiciona(fila, columna);
  pantalla.muestraCursor(true);
  i = 0;
  do {
      caracter = introduceCaracter();
      if (strchr(validos, caracter) != NULL && i < nCaracteres) {
          resultado[i] = caracter;
          i++;
          pantalla.escribeCaracter(caracter);
      }
      if (caracter == 'i') {
          i--;
          pantalla.posiciona(fila, columna+i);
          pantalla.escribeCaracter(' ');
          pantalla.posiciona(fila, columna+i);
      }
  } while (caracter != 'e');
  pantalla.muestraCursor(false);
  resultado[i] = 0;
}

int Consola::introduceEntero(int fila, int columna, int nCaracteres) {
  introduceCadena(fila, columna, "0123456789", nCaracteres, _cadena);
  return atoi(_cadena);
}

float Consola::introduceReal(int fila, int columna, int nCaracteres) {
  introduceCadena(fila, columna, "0123456789.-", nCaracteres, _cadena);
  return atof(_cadena);
}

void Consola::borraPantalla() {
  pantalla.borra();
}
\end{lstlisting}

\subsection{ConnectEsp}
\subsubsection{Cabeceira}
\begin{lstlisting}
#ifndef ConnectEsp_h
#define ConnectEsp_h

#include <Arduino.h>
#include <Ethernet.h>
#include <WiFiEsp.h>
#include <ArduinoJson.h>
#include <stdio.h>


class ConnectEsp
{

private:
	    WiFiEspClient _client;
      //constructor da clase WifiEspClient

	    char *_ssid;
	    char *_pass;
	    //Nome e contraseñas do AP ó que te conectas
	    int _port;
	    char *_host;
	    //Porto e ip do servidor ó que te conectas

	    int status = WL_IDLE_STATUS;
		int pk_encargo;
		int* pk_producto;
		int* cantidade_producto;
		char** name_producto;
		char** localizacion_producto;
		//variables para almacenar request de Json

public:

	    ConnectEsp (char ssid[], char password[], int port, char host[]);
	    // Constructor ó que se lle pasa:
	    // - ssid = nome da rede do AP
	    // - password = contrasinal da rede do AP
	    // - port = porto do host http
	    // - host = direccion do host http

	    void connectAP();
	    // Hay que llamar a este método en la función setup() para conectarse a la red
	    // Devolve True si se conectou co AP con éxito

	    void httpRequest(char message[]);
	    // Hay que llamar a este método en la función setup() para conectarse al servidor
	    // Devolve True si se conectou servidor con éxito
		bool get_Encargo_Completado_django();
		// Función que fai un GET a tabla Encargo e devolve se hai encargos

		int get_producto_Encargo_django();
		//funcion que fai un GET a tabla Detalle_Encargo e devolve o numero de productos para o encargo

		void get_Localizacion_django(int num_producto);
		//función que fai un GET a tabla Producto e lle pasa por parametro o pk do producto

		void post_encargo_completado();
		//función que fai un PUT e modifica o rexitro "Completado" a True

		//getters de variables privadas
		inline int get_pk_Producto(int index){return pk_producto[index];}
		inline char * get_localizacion_producto(int index){return localizacion_producto[index];}
		inline char * get_name_Producto(int index){return name_producto[index];}
		inline int get_cantidade_Producto(int index){return cantidade_producto[index];}
		inline int get_pk_Encargo(){return pk_encargo;}
};


#endif

\end{lstlisting}

\subsubsection{Código}
\begin{lstlisting}
#include "ConnectEsp.h"


ConnectEsp::ConnectEsp(char ssid[], char password[], int port, char host[])
{
  // Constructor ó que se lle pasa:
  // - ssid = nome da rede do AP
  // - password = contrasinal da rede do AP
  // - port = porto do host http
  // - address = direccion do host http (array de bytes)

  _ssid = ssid;
  _pass = password;
  _port = port;
  _host = host;
  // Garda no obxecto estos parametros
}


void ConnectEsp::connectAP()
{
  // En el setup() del sketch hay que llamar a este método para conectarse a la red
  // initialize serial for debugging
  Serial.begin(115200);
  // initialize serial for ESP module
  Serial1.begin(115200);
  // initialize ESP module

    WiFi.init(&Serial1);

    if (WiFi.status() == WL_NO_SHIELD)
    {
      Serial.println("WiFi shield not present");
      // don't continue
      while (true);
    }

    while (status != WL_CONNECTED)
    {
      Serial.print("Conectandose a rede: ");
      Serial.println(_ssid);
      status = WiFi.begin(_ssid,_pass);
    }

    Serial.println("Estás conectado á rede");
}

void ConnectEsp::httpRequest(char message[])
{
  // close any connection before send a new request
  // this will free the socket on the WiFi shield
  _client.stop();
  _client.flush();

  // if there's a successful connection
  if (_client.connect(_host, _port)) {
    Serial.println("Conectando co host...");

    // send the HTTP PUT request
    _client.println(message);
    _client.print("Host: ");
    _client.println(_host);
    _client.println("Connection: close");
    _client.println();
  }
  else
  {
    // if you couldn't make a connection
    Serial.println("Connection failed");
  }
}


bool ConnectEsp::get_Encargo_Completado_django()
{

  while(!_client.available());
  String jsonString;
  while (_client.available())
  {
    jsonString = _client.readStringUntil('\r');
  }
  DynamicJsonBuffer jsonBuffer;

  JsonObject& root = jsonBuffer.parseObject(jsonString);
  if (!root.success()) {
    Serial.println("parseObject() failed");
    //return;
  }
  int count = root["count"];
  JsonArray& results = root["results"];

  for (int i = 0; i < count; i++)
  {
    JsonObject& results_aux = results[i];
    bool completado = results_aux["completado"];
    if(!completado)
    {
      pk_encargo = results_aux["pk"];
      return true;
    }
    return false;
  }
}

int ConnectEsp::get_producto_Encargo_django()
{
  int count_productos_encargo = 0;
  while(!_client.available());
  String jsonString;
  while (_client.available())
  {
    jsonString = _client.readStringUntil('\r');
  }
  Serial.println(jsonString);
  StaticJsonBuffer<700> jsonBuffer;

  JsonObject& root = jsonBuffer.parseObject(jsonString);
  if (!root.success()) {
    Serial.println("parseObject() failed");
    //return;
  }
  int count = root["count"];

  JsonArray& results = root["results"];

  for (int i = 0; i < count; i++)
  {
    JsonObject& results_aux = results[i];
    int pk_encargo_aux = results_aux["encargo"];

    if(pk_encargo_aux == pk_encargo)
    {
      pk_producto[count_productos_encargo] = results_aux["producto"];
      cantidade_producto[count_productos_encargo] = results_aux["cantidade"];;

      count_productos_encargo++;
    }
  }
  return count_productos_encargo;
}

void ConnectEsp::get_Localizacion_django(int num_producto)
{

  while(!_client.available());
  String jsonString;
  while (_client.available())
  {
    jsonString = _client.readStringUntil('\r');
  }
  Serial.println(jsonString);
  StaticJsonBuffer<700> jsonBuffer;

  JsonObject& root = jsonBuffer.parseObject(jsonString);
  if (!root.success()) {
    Serial.println("parseObject() failed");
    //return;
  }

  localizacion_producto[num_producto] = root["localizacion"];
  name_producto[num_producto] = root["name"];
}

void ConnectEsp::post_encargo_completado(/* arguments */)
{

  String content = "{\"completado\":true}";

  if (_client.connect(_host, _port))
  {
    _client.print("PUT /encargos/");
    _client.print(pk_encargo);
    _client.println("/ HTTP/1.0");
    _client.print("Host: ");
    _client.println(_host);
    _client.println("Accept: */*");
    _client.println("Content-Length: " + content.length());
    _client.println("Content-Type: application/x-www-form-urlencoded");
    _client.println();
    _client.println(content);
  }
}

\end{lstlisting}

\section{Código principal (main.cpp)}

\begin{lstlisting}
#include <Arduino.h>
#include "Consola.h"
#include <ConnectEsp.h>
#include <string.h>

//variables estáticas tipo String para a conexión AP
static char ssid[] = "HUAWEI-Cw4a";
static char pass[] = "9RR3VAxS";

//variable estática tipo String do host do servidor
static char host[]= "192.168.100.3";
static int port = 8000;

//declaramos constructor da clase ConnectEsp
ConnectEsp ap (ssid, pass, port, host);

//declaramos constructor da clase Consola
Consola consola(32, 33, 34, 35, 36, 37, 38, 22, 23, 24, 25, 26, 27, 28, 29, "789s456-123d.0ei");

void setup()
{
  // put your setup code here, to run once:
  consola.configura();
  // Configura as señaies Arduino e envía a secuencia de órdenes de inicialización
  // á pantalla e teclado

  ap.connectAP();
  consola.visualizaCadena(-1, 0, "Conectouse a rede: ");
  consola.visualizaCadena(-1, 20, ssid);
  consola.visualizaCadena(4,1,"Pulsa ENT para continuar");

  while(consola.introduceCaracter() != 'e');
  consola.borraPantalla();
}

void loop()
{
    // put your main code here, to run repeatedly:
    ap.httpRequest("GET /encargos/ HTTP/1.0");

    if (ap.get_Encargo_Completado_django())
    {
      consola.visualizaCadena(-1, 0, "Encargo con id: ");
      consola.visualizaEntero(-1, 18, ap.get_pk_Encargo());
      consola.visualizaCadena(4,1,"Pulsa ENT para continuar");

      while(consola.introduceCaracter() != 'e');
      consola.borraPantalla();

      ap.httpRequest("GET /detalle_encargo/ HTTP/1.0");
      int n_p = 0;

      n_p = ap.get_producto_Encargo_django();
      Serial.println(n_p);

      if (n_p > 0)
      {
        for (int i = 0; i < n_p; i++)
        {
          char to_send[30];
          sprintf(to_send,"GET /productos/%d/ HTTP/1.0",2);
          ap.httpRequest(to_send);
          ap.get_Localizacion_django(i);

          consola.visualizaCadena(1, 0, "Producto: ");
          consola.visualizaCadena(1, 10, ap.get_name_Producto(i));
          consola.visualizaCadena(2, 0, "Cantidade: ");
          consola.visualizaCadena(2, 10, ap.get_name_Producto(i));
          consola.visualizaCadena(3, 0, "Localizacion: ");
          consola.visualizaCadena(3, 10, ap.get_localizacion_producto(i));
          consola.visualizaCadena(4,0,"Pulsa ENT cando acabes");

          while(consola.introduceCaracter() != 'e');
          consola.borraPantalla();
        }
        ap.post_encargo_completado();

        consola.visualizaCadena(-1, 0, "Encargo completado");
        consola.visualizaCadena(4,0,"Pulsa ENT para ver o seguinte");

        while(consola.introduceCaracter() != 'e');
        consola.borraPantalla();
      }
      else
      {
        consola.visualizaCadena(-1, 0, "Non hai productos para este encargo");
        consola.visualizaCadena(4,1,"Pulsa ENT para ver o seguinte");

        while(consola.introduceCaracter() != 'e');
        consola.borraPantalla();
      }
    }
    else {
      consola.visualizaCadena(-1, 0, "Non hai encargos");
      consola.visualizaCadena(4,1,"Pulsa ENT para actualizar");

      while(consola.introduceCaracter() != 'e');
      consola.borraPantalla();
    }
}

\end{lstlisting}

\stopcontents[parts]

\cleardoublepage

\renewcommand{\documento}{Pliego de condicións}
\begin{tikzpicture}[remember picture, overlay]
  \draw[line width = 0.5pt] ($(current page.north west) + (-20pt,-800.5pt)$) rectangle ($(current page.south east) + (-133.5pt,-722.5pt)$);
\end{tikzpicture}

\begin{center}
\begin{figure}[htbp]
\begin{center}
\includegraphics[angle=0, height=3.8cm]{images/EEILogo.png}
\end{center}
\end{figure}
\ \\
\begin{large}
\begin{center}
\color{blue}\textbf{Escola de Enxeñería Industrial}
\end{center}
\end{large}
\ \\
\ \\
\begin{large}
\begin{center}
\textbf{TRABALLO FIN DE GRAO}
\end{center}
\end{large}
\ \\
\ \\
\begin{large}
\begin{center}
{\titulouno}
\end{center}
\end{large}
\ \\
\ \\
\begin{normalsize}
\begin{center}
\textbf{\grado}
\end{center}
\end{normalsize}
\ \\
\ \\
\ \\
\ \\
\begin{normalsize}
\begin{center}
\textbf{Documento}
\end{center}
\end{normalsize}
\ \\
\begin{normalsize}
\begin{center}
\part{\bf{PLIEGO DE CONDICIÓNS}}\thispagestyle{empty}
\end{center}
\end{normalsize}
\ \\
\ \\
\ \\
\ \\

\begin{center}
\begin{figure}[htbp]
\begin{center}
\includegraphics[angle=0, height=0.8cm]{images/UVIGOLogo.png}
\end{center}
\end{figure}
\end{center}

\end{center}

\cleardoublepage


\pagestyle{fancy}
%%%%%%%%%%%%%%%%%%%%%%%%%%%%%%%%%%%%%%%%%%%%%%%%%%%%%%%%%%%%%%%%%%%%%%%%%%%%%%%%%%%%%%%%%%%%%%%%%%%%%%%%%%%%%%%%%%%%%%%%%%%%%%%%%%%%%%%%%%%%%%%%%%%
%%%%%%%%%%%%%%%%%%%%%%%%%%%%%%%%%%%%%%%%%%%%%%%%%%%%%%%%%%%%%%%%%%%%%%%%%%%%%%%%%%%%%%%%%%%%%%%%%%%%%%%%%%%%%%%%%%%%%%%%%%%%%%%%%%%%%%%%%%%%%%%%%%%%
\addcontentsline{toc}{section}{Índice del documento Pliego de condiciones}
\startcontents[parts]
\begin{center}{\large \bf Índice do documento PLIEGO DE CONDICIÓNS}\end{center}

{\hypersetup{hidelinks}\printcontents[parts]{}{-1}{\setcounter{tocdepth}{5}}}

\cleardoublepage

\chapter{Obxeto do Pliego de Condicións}

O obxeto deste documento é establecer os criterios para a relación establecida entre os axentes implicados nas obras definidas neste proxecto e servir de base para a execución do contrato entre o Enxeñeiro Director e a empresa demandate.

O obxeto deste documento é establecer as condicións técnicas e facultativas que van servir de base para a execución deste proxecto.

As condicións que se detallarán a continuación tratan de cumplir coa calidade esperada. No caso de non realizarse coas condicións reflexadas, o Enxeñeiro Director non se fai cargo dos fallo ou averías que poidan ocurrir no seu funcionamento.


\chapter{Disposición de carácter xeral}

Este proxecto axústase coas normativas electrónicas vixentes. Os cambios ou modificacións levadas a cabo unha vez rematado o proxecto so se poderían facer baixo a supervisión do Enxeñerio Xefe.

A propiedade intelectual do autor e director do TFG rexirase polo Real Decreto Lexislativo 1/1996, do 12 de abril.

\section{Normativa a seguir}
A normativa a seguir vai ser:
\begin{itemize}
    \item Reglamento Electrotécnico para baixa tensión.
    \item Norma UNE 21302-2: 1973: Vocabulario electrotécnico.
    \item Norma UNE-EN 61000-4-3-1998: Compatibilidade electromagnética.
    \item Norma UNE-EN 60439-1: Requisitos para cables e conductores.
    \item Norma UNE 20-050-74: Códigos para marcas de resistencia e condensadores.
    \item Norma UNE 20-524-75: Técnica de circuitos impresos.
    \item Norma UNE 20-501-86: Equipos electrónicos e compoñentes.
    \item Real Decreto 1215/1997: Normativa de Seguridade e Saúde no traballo.
\end{itemize}

\section{Pago e entrega do traballo}
\chapter{Condicións particulares}

\begin{itemize}
    \item Todos os traballo serán realizados por técnicos preparados.
    \item Todas as dudas que surxan despois da súa realización poderán ser resoltas polo Enxeñeiro Xefe.
    \item O instalador poderá requerir do Enxeñeiro Xefe 
\end{itemize}

\stopcontents[parts]
\cleardoublepage
%%%%%%%%%%%%%%%%%%%%%%%%%%%%%%%%%%%%%%%%%%%%%%%%%%%%%%%%%%%%%%%%%%%%%%%%%%%%%%%%%%%%%%%%%%%%%%%%%%%%%%%%%%%%%%%%%%%%%%%%%%%%%%%%%%%%%%%%%%%%%%%%%%%%%%%%%%%%%%%%%%%%%%%%%%%%%
%%%%%%%%%%%%%%%%%%%%%%%%%%%%%%%%%%%%%%%%%%%%%%%%%%%%%%%%%%%%%%%%%%%%%%%%%%%%%%%%%%%%%%%%%%%%%%%%%%%%%%%%%%%%%%%%%%%%%%%%%%%%%%%%%%%%%%%%%%%%%%%%%%%%%%%%%%%%%%%%%%%%%%%%%%%%
% DOCUMENTO PRESUPUESTO
\pagestyle{empty}
\renewcommand{\documento}{ORZAMENTO}
\begin{tikzpicture}[remember picture, overlay]
  \draw[line width = 0.5pt] ($(current page.north west) + (-20pt,-800.5pt)$) rectangle ($(current page.south east) + (-133.5pt,-722.5pt)$);
\end{tikzpicture}

\begin{center}
\begin{figure}[htbp]
\begin{center}
\includegraphics[angle=0, height=3.8cm]{images/EEILogo.png}
\end{center}
\end{figure}
\ \\
\begin{large}
\begin{center}
\color{blue}\textbf{Escola de Enxeñería Industrial}
\end{center}
\end{large}
\ \\
\ \\
\begin{large}
\begin{center}
\textbf{TRABALLO FIN DE GRAO}
\end{center}
\end{large}
\ \\
\ \\
\begin{large}
\begin{center}
{\titulouno}
\end{center}
\end{large}
\ \\
\ \\
\begin{normalsize}
\begin{center}
\textbf{\grado}
\end{center}
\end{normalsize}
\ \\
\ \\
\ \\
\ \\
\begin{normalsize}
\begin{center}
\textbf{Documento}
\end{center}
\end{normalsize}
\ \\
\begin{normalsize}
\begin{center}
\part{\bf{ORZAMENTO}}
\end{center}
\end{normalsize}
\ \\
\ \\
\ \\
\ \\

\begin{center}
\begin{figure}[htbp]
\begin{center}
\includegraphics[angle=0, height=0.8cm]{images/UVIGOLogo.png}
\end{center}
\end{figure}
\end{center}

\end{center}

\cleardoublepage


\pagestyle{fancy}
%%%%%%%%%%%%%%%%%%%%%%%%%%%%%%%%%%%%%%%%%%%%%%%%%%%%%%%%%%%%%%%%%%%%%%%%%%%%%%%%%%%%%%%%%%%%%%%%%%%%%%%%%%%%%%%%%%%%%%%%%%%%%%%%%%%%%%%%%%%%%%%%%%%%%%%%%%%%%%%%%%%%%%%%%%%%%
%%%%%%%%%%%%%%%%%%%%%%%%%%%%%%%%%%%%%%%%%%%%%%%%%%%%%%%%%%%%%%%%%%%%%%%%%%%%%%%%%%%%%%%%%%%%%%%%%%%%%%%%%%%%%%%%%%%%%%%%%%%%%%%%%%%%%%%%%%%%%%%%%%%%%%%%%%%%%%%%%%%%%%%%%%%%
\addcontentsline{toc}{section}{Índice del documento Orzamento}
\startcontents[parts]
\begin{center}{\large \bf Índice do documento ORZAMENTO}\end{center}

{\hypersetup{hidelinks}\printcontents[parts]{}{-1}{\setcounter{tocdepth}{5}}}

\cleardoublepage


% Incluir contenido PRESUPUESTO


\chapter{Orzamento parcial}

\section{Planificación}

\begin{table}[htbt]
\begin{center}
    \begin{tabular}{|rrrlr|}
    \toprule
    \rowcolor[rgb]{ .31,  .506,  .741} \multicolumn{1}{|l}{\textcolor[rgb]{ 1,  1,  1}{\textbf{Descripción}}} & \multicolumn{1}{l}{\textcolor[rgb]{ 1,  1,  1}{\textbf{Ud}}} & \multicolumn{1}{l}{\textcolor[rgb]{ 1,  1,  1}{\textbf{Cantidade}}} & \textcolor[rgb]{ 1,  1,  1}{\textbf{Precio unitario}} & \multicolumn{1}{l|}{\textcolor[rgb]{ 1,  1,  1}{\textbf{Precio total}}} \\
    \midrule
    \rowcolor[rgb]{ .863,  .902,  .945} \multicolumn{1}{|l}{Desplazamentos} & \multicolumn{1}{c}{Viaxes} & \multicolumn{1}{c}{30} & \multicolumn{1}{r}{0.85 \euro} & 25.50 \euro \\
    \midrule
    \multicolumn{1}{|l}{Documentación} & \multicolumn{1}{c}{h} & \multicolumn{1}{c}{60} & \multicolumn{1}{r}{10.00 \$} & 600.00 \euro \\
    \midrule
    \rowcolor[rgb]{ .863,  .902,  .945} \multicolumn{1}{|l}{Internet} & \multicolumn{1}{c}{h} & \multicolumn{1}{c}{80} & \multicolumn{1}{r}{0.60 \$} & 48.00 \$ \\
    \midrule
    \multicolumn{1}{|l}{Uso do ordenador} & \multicolumn{1}{c}{h} & \multicolumn{1}{c}{100} & \multicolumn{1}{r}{0.30 \$} & 30.00 \euro \\
    \midrule
    \rowcolor[rgb]{ .863,  .902,  .945}       &       &       & Subtotal & 703.50 \euro \\
    \midrule
          &       &       & IVA 21\% & 147.74 \$ \\
    \midrule
    \rowcolor[rgb]{ .863,  .902,  .945}       &       &       & Total & 851.24 \euro \\
    \bottomrule
    \end{tabular}%
\caption{Prezos planificación}
\label{PrezosPlanificacion}
\end{center}
\end{table}

\section{Desenvolvemento do proxecto}

\begin{table}[htbt]
\begin{center}
    \begin{tabular}{|rrrlr|}
    \toprule
    \rowcolor[rgb]{ .31,  .506,  .741} \multicolumn{1}{|l}{\textcolor[rgb]{ 1,  1,  1}{\textbf{Descripción}}} & \multicolumn{1}{l}{\textcolor[rgb]{ 1,  1,  1}{\textbf{Ud}}} & \multicolumn{1}{l}{\textcolor[rgb]{ 1,  1,  1}{\textbf{Cantidade}}} & \textcolor[rgb]{ 1,  1,  1}{\textbf{Precio unitario}} & \multicolumn{1}{l|}{\textcolor[rgb]{ 1,  1,  1}{\textbf{Precio total}}} \\
    \midrule
    \rowcolor[rgb]{ .863,  .902,  .945} \multicolumn{1}{|l}{Desplazamentos} & \multicolumn{1}{c}{Viaxes} & \multicolumn{1}{c}{180} & \multicolumn{1}{r}{0.85 \euro} & 153.00 \euro \\
    \midrule
    \multicolumn{1}{|l}{Diseño} & \multicolumn{1}{c}{h} & \multicolumn{1}{c}{450} & \multicolumn{1}{r}{10.00 \euro} & 4,500.00 \euro \\
    \midrule
    \rowcolor[rgb]{ .863,  .902,  .945} \multicolumn{1}{|l}{Internet} & \multicolumn{1}{c}{h} & \multicolumn{1}{c}{150} & \multicolumn{1}{r}{0.60 \euro} & 90.00 \euro \\
    \midrule
    \multicolumn{1}{|l}{Uso do ordenador} & \multicolumn{1}{c}{h} & \multicolumn{1}{c}{500} & \multicolumn{1}{r}{0.30 \euro} & 150.00 \euro \\
    \midrule
    \rowcolor[rgb]{ .863,  .902,  .945}       &       &       & Subtotal & 4,893.00 \euro \\
    \midrule
          &       &       & IVA 21\% & 1,027.53 \euro \\
    \midrule
    \rowcolor[rgb]{ .863,  .902,  .945}       &       &       & Total & 5,920.53 \euro \\
    \bottomrule
    \end{tabular}%
\caption{Prezos desenvolvemento do proxecto}
\label{PrezosDesenvolvementoProxecto}
\end{center}
\end{table}

\begin{table}[htbt]
\begin{center}
    \begin{tabular}{|rrrlr|}
    \toprule
    \rowcolor[rgb]{ .31,  .506,  .741} \multicolumn{1}{|l}{\textcolor[rgb]{ 1,  1,  1}{\textbf{Descripción}}} & \multicolumn{1}{l}{\textcolor[rgb]{ 1,  1,  1}{\textbf{Referencias}}} & \multicolumn{1}{l}{\textcolor[rgb]{ 1,  1,  1}{\textbf{Cantidade}}} & \textcolor[rgb]{ 1,  1,  1}{\textbf{Precio unitario}} & \multicolumn{1}{l|}{\textcolor[rgb]{ 1,  1,  1}{\textbf{Precio total}}} \\
    \midrule
    \rowcolor[rgb]{ .863,  .902,  .945} \multicolumn{1}{|l}{Placa} & \multicolumn{1}{c}{Arduino Mega 2560} & \multicolumn{1}{c}{1} & \multicolumn{1}{r}{35.00 \euro} & 35.00 \euro \\
    \midrule
    \multicolumn{1}{|l}{Teclado} & \multicolumn{1}{c}{Keypad 4x4 Storm} & \multicolumn{1}{c}{1} & \multicolumn{1}{r}{25.00 \euro} & 25.00 \euro \\
    \midrule
    \rowcolor[rgb]{ .863,  .902,  .945} \multicolumn{1}{|l}{Pantalla} & \multicolumn{1}{c}{LCD Midas 40x4} & \multicolumn{1}{c}{1} & \multicolumn{1}{r}{40.00 \euro} & 40.00 \euro \\
    \midrule
    \multicolumn{1}{|l}{Módulo Wifi} & \multicolumn{1}{c}{ESP8266} & \multicolumn{1}{c}{1} & \multicolumn{1}{r}{5.00 \euro} & 5.00 \euro \\
    \midrule
    \rowcolor[rgb]{ .863,  .902,  .945} \multicolumn{1}{|l}{Protoboard} &       & \multicolumn{1}{c}{1} & \multicolumn{1}{r}{12.00 \euro} & 12.00 \euro \\
    \midrule
    \multicolumn{1}{|l}{Schneider Electronics} & \multicolumn{1}{c}{Modicon Premium TSX} & \multicolumn{1}{c}{1} & \multicolumn{1}{r}{996.83 \euro} & 996.83 \euro \\
    \midrule
          &       &       & Subtotal & 1,113.83 \euro \\
    \midrule
    \rowcolor[rgb]{ .863,  .902,  .945}       &       &       & IVA 21\% & 233.90 \euro \\
    \midrule
          &       &       & Total & 1,347.73 \euro \\
    \bottomrule
    \end{tabular}%
\caption{Táboa prezos de compoñentes}
\label{PrezosComponentes}
\end{center}
\end{table}

\begin{table}[htbt]
\begin{center}
    \begin{tabular}{|rrrlr|}
    \toprule
    \rowcolor[rgb]{ .31,  .506,  .741} \multicolumn{1}{|l}{\textcolor[rgb]{ 1,  1,  1}{\textbf{Descripción}}} & \multicolumn{1}{c}{\textcolor[rgb]{ 1,  1,  1}{\textbf{Ud}}} & \multicolumn{1}{c}{\textcolor[rgb]{ 1,  1,  1}{\textbf{Cantidade}}} & \textcolor[rgb]{ 1,  1,  1}{\textbf{Precio unitario}} & \multicolumn{1}{l|}{\textcolor[rgb]{ 1,  1,  1}{\textbf{Precio total}}} \\
    \midrule
    \rowcolor[rgb]{ .863,  .902,  .945} \multicolumn{1}{|l}{Copistería} & \multicolumn{1}{c}{Viaxes} & \multicolumn{1}{c}{1} & \multicolumn{1}{r}{60.00 \euro} & 60.00 \euro \\
    \midrule
    \multicolumn{1}{|l}{Horas de traballo} & \multicolumn{1}{c}{h} & \multicolumn{1}{c}{200} & \multicolumn{1}{r}{10.00 \euro} & 2,000.00 \euro \\
    \midrule
    \rowcolor[rgb]{ .863,  .902,  .945} \multicolumn{1}{|l}{Internet} & \multicolumn{1}{c}{h} & \multicolumn{1}{c}{100} & \multicolumn{1}{r}{0.60 \euro} & 60.00 \euro \\
    \midrule
    \multicolumn{1}{|l}{Uso do ordenador} & \multicolumn{1}{c}{h} & \multicolumn{1}{c}{100} & \multicolumn{1}{r}{0.30 \euro} & 30.00 \euro \\
    \midrule
    \rowcolor[rgb]{ .863,  .902,  .945}       &       &       & Subtotal & 2,150.00 \euro \\
    \midrule
          &       &       & IVA 21\% & 451.50 \$ \\
    \midrule
    \rowcolor[rgb]{ .863,  .902,  .945}       &       &       & Total & 2,601.50 \euro \\
    \bottomrule
    \end{tabular}%
\caption{Prezos da redacción do proxecto}
\label{PrezosRedaccion}
\end{center}
\end{table}

\chapter{Orzamento total}

\begin{table}[htbt]
\begin{center}
    \begin{tabular}{|lr|}
    \toprule
    \rowcolor[rgb]{ .31,  .506,  .741} \textcolor[rgb]{ 1,  1,  1}{\textbf{Orzamento}} & \multicolumn{1}{l|}{\textcolor[rgb]{ 1,  1,  1}{\textbf{Precio total}}} \\
    \midrule
    \rowcolor[rgb]{ .863,  .902,  .945} Planificación & 851.24 \euro\\
    \midrule
    Desenvolvemento do proxecto & 5,920.53 \euro \\
    \midrule
    \rowcolor[rgb]{ .863,  .902,  .945} Compoñentes & 1,347.73 \euro \\
    \midrule
    Redacción do proxecto & 2,601.50 \euro \\
    \midrule
    \rowcolor[rgb]{ .863,  .902,  .945} \textbf{TOTAL} & 10,721.00 \euro \\
    \bottomrule
    \end{tabular}%
\caption{Prezo total do orzamento}
\label{PrezoTotal}
\end{center}
\end{table}

O IMPORTE TOTAL DO DESENVOLVEMENTO DO PRESENTE PROXECTO ASCENDE A DOCEMIL CINCOCENTOS TRINTA E SEIS EUROS (12.536 \euro).
\\
\\
\\
\\
\\
Álvaro Fernández Quesada \\
76731592-G
\\
\\
\\
\\
\\
\\
Vigo a 5 de Xuño de 2018

\stopcontents[parts]

\cleardoublepage

\end{document}

