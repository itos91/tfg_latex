%%%%%%%%%%%%%%%%%%%%%%%%%%%%%%%%%%
% Página de resumen del proyecto %
%%%%%%%%%%%%%%%%%%%%%%%%%%%%%%%%%%

\thispagestyle{empty}

\bigskip
\bigskip

\large{
\textbf{Resumo:}}

\bigskip
\bigskip


\begin{center}
\textbf{\titulouno}
\end{center}

\bigskip
\bigskip
%\bigskip
\large{
\textbf{Alumno:}}\alumno

%\medskip
\large{
\textbf{Director:}} \tutoruno

%\vfill

%\begin{minipage}{\textwidth}
%\textbf{Dpto. de:}
%


%\medskip
%
%\textbf{Titulación:} Ingeniería de Telecomunicación
%
%\medskip
\bigskip
\bigskip


O presente proxecto lévase a cabo a implementación de un sistema de xestión de preparación de pedidos de varios productos nun almacén.

Desarrollouse un programa elaborado en C++ para unha placa Arduino para que sirva como terminal de operador para un carretilleiro de un almacén e que desde o cal podan xestionar encargos comunicándose por WiFi a un servidor.

O terminal de operador está composto por unha pantalla LCD para visualizar en todo momento o encargo e por un teclado matricial para indicar que o producto depositouse nunha caixa.
A comunicación por WiFi é levada a cabo polo módulo ESP8266 mediante servicios REST a un servidor que dispón dunha base de datos relacional.

O entorno de desenvolvemento dos servicios REST fixéronse con un framework de python chamado API RESTful Django e o programa para Arduino, por Atom.
 

\bigskip
\bigskip

\textbf{Palabras clave:} IoT, Picking, Almacén, Arduino, Servicios RESTful, Python.

%\begin{center} Vigo, \today\end{center}
%\end{minipage}

%Página en blanco
\newpage{\pagestyle{empty}\cleardoublepage}